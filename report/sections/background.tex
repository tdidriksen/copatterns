%!TEX root = ../main.tex
\section{Background}
\label{sec:background}

\subsection{Codata}
What is codata?

\subsection{Copatterns}
What are copatterns? (conceptually)
% Copatterns in Agda

% Inductively defined data can be analyzed by defining functions on that data by traditional pattern matching, where a \emph{pattern} is a (finite and valid) combination of constructor applications for the given data. In this respect, 

% \begin{figure}
% \begin{alltt}
% pow2 : Stream Nat
% head pow2 = S Z
% head (tail pow2) = S (S Z)
% tail (tail pow2) = zipWith _+_ (tail pow2) (tail pow2)
% \end{alltt}
% \caption{A definition of an infinite stream of powers of 2 using copatterns.}
% \end{figure}

% by defining functions on that data using traditional pattern 

% Inductively defined data can be analyzed by defining functions in terms of a set of \textit{patterns}, where a pattern constitutes a recognizable structure in the data we are analyzing. 




% Because we know that the data has been constructed using a finite number of data constructors, the number of ways we can take that same data apart is also finite by definition. 


% Consider a standard list data structure, defined in Haskell-like syntax:

% \begin{alltt}
% data List a = Nil 
%             | Cons a (List a)
% \end{alltt}

% \texttt{List a} consists of two data constructors, \texttt{Nil} and \texttt{Cons}. Because these two constructors constitute the only that we can construct a \texttt{List a}, 

\subsection{The Current Status of Codata in Idris}