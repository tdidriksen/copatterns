 %!TEX TS-options = -shell-escape
\documentclass[oribibl]{llncs}
\pagestyle{headings}
\usepackage{natbib}
\bibliographystyle{alpha}
\newenvironment{changemargin}[2]{%
\begin{list}{}{%
\setlength{\topsep}{0pt}%
\setlength{\leftmargin}{#1}%
\setlength{\rightmargin}{#2}%
\setlength{\listparindent}{\parindent}%
\setlength{\itemindent}{\parindent}%
\setlength{\parsep}{\parskip}%
}%
\item[]}{\end{list}}

\usepackage[utf8]{inputenc}
\usepackage{alltt}
\usepackage{url}
\usepackage{todonotes}
\usepackage{fixltx2e} % for subscript
\usepackage{fancyvrb}
\usepackage{amsmath}

\newcommand{\Idris}{\textsc{Idris}}
\newcommand{\IdrisM}{\textsc{Idris}$^-$}
\newcommand{\TT}{\textsf{TT}}

\begin{document}
\mainmatter
\title{Understanding and Implementing Copatterns}
\author{Sune Alk\ae{}rsig and Thomas Hallier Didriksen \\
\email{\{sual, thdi\}@itu.dk}}

\institute{IT University of Copenhagen, Rued Langgaards Vej 7, 2300 Copenhagen S, Denmark}
\maketitle

\begin{abstract}
While inductive data can be understood in terms of constructors or introduction rules, coinductive data can be understood in terms of observations or elimination rules. We explore the idea of programming by observations, where coinductive definitions are defined with \emph{copatterns}, a construct for defining the results of observations. Through the implementation of a small functional language with coinductive types and copatterns, we investigate the underlying properties of this construct. Furthermore, after providing a survey of existing approaches to productivity checking, we devise a productivity checking algorithm for definitions with copatterns which is based on a purely syntactic check. These efforts serve as preparatory work for our coming master's thesis, where we plan to implement copatterns in Idris. Therefore, an analysis of the needed changes to the Idris compiler is also provided.
\keywords{Copatterns, Coinductive data, Corecursion, Idris, Totality}
\end{abstract}

%!TEX root = ../main.tex
\section{Introduction}
\label{sec:introduction}
%Copatterns. This is an preparatory study for our master's thesis.
%How do we fit copatterns into Idris?

In the world of functional programming, the use of inductive data types for data modeling has long been a standard practice. Inductive data types enable users to model fundamental data structures such as lists, trees, and graphs, and to analyze such data by pattern matching. Common to all inductively defined data is that it is inherently finite. Since it is defined in terms of constructors, infinite structures cannot be defined inductively in systems with finite memory unless certain evaluation strategies are abandoned. In particular, programming languages such as Haskell has successfully been able to blur the lines between finite and infinite structures by leveraging the power of lazy evaluation. Nevertheless, a more elegant approach to modeling infinite data is by the use of \emph{co}inductive data types. These were first implemented by Hagino for his SymML language\,\citep{Hagino89}, allowing users to define coinductive data in terms of its \emph{destructors}. This means that it does not make sense to think of coinductive data as something that is constructed. Rather, we think of coinductive data as something we can observe, in the sense that we can observe the values of the destructors at any given time. Coinductive data types have since been added to several other programming systems, examples of which are the Coq proof system\,\citep{Coq:manual}, Agda\,\citep{Norell:thesis}, and Idris\,\citep{BradyIdrisImpl13}.

While most (if not all) systems with inductive data types have a pattern matching construct for analyzing data, many systems with coinductive data types do not have dedicated constructs for manipulating destructors. Instead, coinductive data is also subjected to pattern matching. In SymML, Hagino describes a \texttt{merge} construct for describing the values of the destructors in a given coinductive definition. In Coq, no such construct exists, and the same is the case for Idris. In the wake of work done by Abel and Pientka\,\citep{Abel13Wellfounded} and Abel, Pientka, Thibodeau, and Setzer\,\citep{Abel13Copatterns}, a construct similar to Hagino's \texttt{merge} has recently been implemented in Agda. This construct is called a \emph{copattern}. By allowing projections on the left-hand side of function definitions, copatterns enable users to define corecursive functions in terms of observations (used interchangeably with ``destructors''). In effect, coinductive data no longer has to be analyzed with pattern matching, but can more naturally be expressed by the observations we can make on it. 

\subsection{Goals and Contributions}
This project serves as a preparation for our master's thesis, where we plan to implement copatterns and coinductive data defined by observations in Idris. Therefore, the goals of this project are:

\begin{itemize}
\item To understand the ideas behind copatterns.
\item To understand how copatterns are used in practice, for example by studying Agda.
\item To understand how copatterns can be implemented.
\item To implement a simple functional programming language with copatterns.
\item To understand how the current termination checker in Idris works by implementing a similar termination checker employing the size-change principle\,\citep{LeeJones01SizeChange} for said language.
\end{itemize}

As a result of pursuing these goals, our contributions are:

\begin{itemize}
\item An investigation into whether the intended changes can be implemented in Idris without changing its core type theory.
\item A survey of the literature describing different implementations of the size-change principle. This is done in order to find out whether it is feasible to extend the size-change termination checker in Idris to work as a productivity checker for corecursive function definitions with copatterns.
\item A proposal for a productivity checking algorithm for corecursive function definitions with copatterns.
\end{itemize} 

\subsection{Outline}
In Section~\ref{sec:background}, we describe the core concepts behind coinductive data, copatterns, and productivity. Section~\ref{sec:related_work} provides a survey of the relevant literature, in particular with respect to termination and productivity checking. The implementation of a small functional language with copatterns is described in Section~\ref{sec:implementing-copatterns}. In Section~\ref{sec:productivity}, we present our proposal for a productivity algorithm for corecursive functions with copatterns, while also discussing the virtues of other approaches to productivity checking. Section~\ref{sec:copatterns_in_idris} explains the necessary changes in order for us to add definitions with copatterns to Idris, and finally, Section~\ref{sec:conclusion} provides a conclusion on the entire project.

%!TEX root = ../main.tex
\section{Background}
\label{sec:background}

\subsection{Codata}

\subsection{Copatterns}

\subsection{Idris Internals}
%!TEX root = ../main.tex
\section{Related Work}
\label{sec:related_work}
This section gives a survey of relevant literature, mostly covering approaches to ensuring termination or productivity of programs defined over both inductive and coinductive data.

\subsection{Size-change Termination}
Due to the fact that Idris already has a working totality checker using size-change termination\,\citep{BradyIdrisImpl13}, it is important that a solution for determining the productivity of corecursive functions takes this principle into account. The size-change principle for termination was first proposed for a strict first-order functional language (without loop constructs) by Lee, Jones, and Ben-Amram\,\citep{LeeJones01SizeChange}. The principle essentially states that if infinitely many recursive calls to a function would lead to infinite decrease in some parameter value, then the function must be terminating, since any value of an inductive type must have finite size. This last condition is of particular importance, as the size-change principle cannot in general recognize functions as being terminating if they have parameters that do not exhibit a well-founded order. Lee, Jones, and Ben-Amram present two realizations of the principle, one using automata and one using a call graph. In the graph formulation, termination is determined by identifying any recursive calls (both direct and indirect) through cycles in the call graph, and then constructing a ``size-change graph'' for each of these. The size-change graphs are then used to find out whether infinite descent in some parameter value is present. One of the limitations of this approach is that parameter values must decrease monotonically: Values cannot at any point become structurally larger, even though the total change in size in a call chain would ultimately lead a to a decreasing value.

Since its first-order formulation, the principle has been proven to work for more expressive cases. Jones and Bohr\,\citep{Jones04Untyped} showed that size-change termination can be applied to the untyped lambda calculus using abstract interpretation. Interesting results of this work include size-change termination of programs employing the Y-combinator.

Following the work on the untyped lambda calculus, Sereni and Jones generalized the size-change principle to handle a higher-order functional language with user-defined data types and general recursion\,\citep{Sereni05terminationanalysis,Sereni06Phd}. Here, a termination criterion is presented which works for arbitrary control-flow graphs, and in turn is able to give an approximation of termination for lazy functional programs. A key point in this work is how different approaches to control-flow and call graph construction may influence the preciseness of the termination analysis.

Most implementations of the size-change principle only work on data for which some well-founded order exists. Nevertheless, Avery\,\citep{Avery06} presented a formulation in which it is possible to detect size-change termination for non-well-founded data types --- in particular, this formulation is shown to work for a language with an integer type. Instead of identifying infinite descent using a well-founded partial order on parameters, the analysis is based on a decrease in invariants which are found to hold for each program point. These invariants are inferred automatically from the structure of the program. The idea is that if the value of some invariant (which can involve arbitrarily many values) can be shown to decrease on every passage of a program point, then the program terminates. While an approach to size-change termination involving non-well-founded data is closer to a usable productivity algorithm for corecursive functions with copatterns, Avery's formulation is insufficient or impractical in several aspects. First, it is still assumed that totality is determined solely on the basis of an infinitely decreasing property that is bounded from below. However, many corecursive functions exhibit no such property; Indeed, one of the virtues of such functions is exactly the ability to define infinite structures. Secondly, the formulation works on a graph of program points. Since it is valid for a corecursive function to be defined in terms of itself, non-termination for copatterns only becomes apparent when observations are defined in terms of themselves. Constructing a graph of all observations in the system would be necessary for this approach to work, and such an undertaking might not be ideal, since it would require all observations to be unfolded (see Section \ref{sec:productivity}).

In more recent work, Hyvernat\,\citep{Hyvernat13} has proposed a formulation of the size-change principle for functional languages which to a certain degree solves the problem of non-monotonic decrease in parameter values. The motivation behind this work is to incorporate size-change termination into the PML language\,\cite{PMLLanguage}. Non-monotonic decrease is detected by tracking the size of a parameter throughout the entire control-flow graph, instead of merely recording whether each call in isolation leads to a decrease in some value. As an extension, Hyvernat proposes that infinite data can be covered by the principle by counting the number of (lambda) abstractions and compare these to the number of function applications in a program. In this form, however, this proposal seems inadequate for a usable productivity checker for copatterns, since it relies directly on function abstractions.

The termination criterion that comes closest to the size-change principle, and which actually predates the original size-change article (\citep{LeeJones01SizeChange}), is the one developed for the \texttt{foetus} termination checker by Abel\,\citep{Abel98foetus}. This criterion forms the basis of the totality checker used by Agda\,\citep{Norell:thesis}. In a similar manner, Abel identifies recursive calls in a call graph and determines termination by tracking changes in parameter sizes. This work does not make any mention of productivity for coinductive data, however.

It seems that there exists no current formulation of the size-change principle which describes how productivity for coinductive data can be ensured in a way that is suitable for a system with copatterns. In the following, we will explore several different approaches to productivity, unrelated to the size-change principle.

\subsection{Guarded Corecursion}
The basic idea of guarded corecursion is that the productivity of corecursive functions can be ensured by a purely syntactic check. It was first proposed by Coquand\,\citep{Coquand94} as an important part of a ``guarded proof induction principle'' for a proof system containing coinductive definitions, inspired by a similar approach in the area of process calculi by Milner\,\citep{Milner82}. In continuation of Coquand's efforts, Gim\'{e}nez applied the method to the Calculus of Constructions in order to avoid the introduction of non-normalizable terms\,\citep{Gimenez95}. 

The guardedness property described by Coquand and Gim\'{e}nez was intended to be used within a proof system (e.g. Coq\,\citep{Coq:manual}) in order to prove specifications involving coinductive types, not necessarily taking a more practical programming setting into account. In such setting, Telford and Turner argue that this approach is too conservative\,\citep{Telford98ensuringthe}. Because \emph{all} corecursive calls to a function must appear directly under a constructor, many intuitive defintions of standard functions (e.g. \texttt{nats} in Figure \ref{fig:nats}) are not considered productive.

In their \emph{Elementary Strong Functional Programming} (ESFP) system\,\citep{Telford97ensuringstreams,Telford98ensuringthe}, Telford and Turner extend the principle to accept a wider range of corecursive functions as being productive, showing it to be usable in a realistic programming setting. They achieve this by considering guardedness more abstractly over a domain of guardedness levels, such that productive corecursion is bounded by a given ``depth''. As an advanced example, they show that the Hamming function is considered productive within their system. A variant of this idea will be discussed in relation to copatterns in Section \ref{sec:productivity}\todo{Remember this forward reference to Section \ref{sec:productivity}}.

Another approach to coping with the conservative nature of the original guardedness principle is to consider alternative programming styles, making productive definitions easier to write. Following the incorporation of the guardedness condition into the Agda totality checker\,\citep{AltenkirchNAD10}, Danielsson\,\citep{Danielsson10beatingthe} described a method for working around the guardedness condition whenever it rejects a productive program. For each situation, he designs an embedded domain specific language, implements the rejected program in said language, and then provides an interpreter which is accepted by the guardedness condition. None of these steps happen automatically, but must be done manually. Although useful, Danielsson argues that efficiency is a concern, and that the best solution might be to entirely move away from using guardedness for productivity. For Agda, sized types have since been implemented.


\subsection{Sized Types}
\label{sec:sized_types}
The idea of sized types is that instead of detecting totality using a syntactic check, such as it is the case for guarded corecursion, size information is added to the type system. Because the type system can then provide guarantees about the sizes of both the input and output of a function, this enables the user to write stronger type specifications. The term ``size'' is used quite broadly here, without implying any specific structure of the data. Unlike the size-change principle or guarded corecursion, approaches using sized types for totality checking of both inductive and coinductive data within the same system have been presented\,\citep{Abel13Wellfounded}.

Sized types for totality checking were first proposed by Hughes, Pareto, and Sabry\,\citep{Hughes96} for reactive systems, where each data type introduced into a program is associated with a family of sized types indicating their bounds. A similar idea was developed by Amadio and Coupet-Grimal\,\citep{Amadio98}, where guard conditions are introduced into the type system to ensure the productivity of coinductive data, following the work of Coquand\,\citep{Coquand94} and Gim\'{e}nez\,\citep{Gimenez95}.

Eduarde Gim\'{e}nez also presented a system for typing recursive definitions in an extension of the Calculus of Constructions using sized types\,\citep{Gimenez98structuralrecursive}. A notable result of this work is that any well-typed term in the proposed extension is normalizing with respect to lazy evalution, widening the domain of functions to which type-based termination is applicable substantially. In the wake of this extension, Abel\,\citep{Abel99terminationchecking} wrote a quite accessible paper on using sized types for termination checking of both inductive types, showing that bidirectional type checking\,\citep{Pierce00} is suitable for a system with sized types. In addition, it is described how infinite streams defined by coinductive types can be encoded in a language as functions on natural numbers, and that the productivity of a stream can be understood in terms of its \emph{definedness}, meaning the number of times it can safely be unfolded.

The notion of definedness is also an important part of Abel and Pientka's work on applying sized types to a system with copatterns\,\citep{Abel13Wellfounded}, although here it is defined more precisely as the \emph{depth} of a value of coinductive type. By directly using the notion of depth in the type system, they show that totality of both coinductive definitions with copatterns and inductive definitions (as well as mixed inductive-coinductive) can be determined by well-founded induction on sizes within the type system. The method is shown to work for System F\textsubscript{$\omega$}, and has later been implemented in Agda along with copatterns. As such, sized types are a candidate for detecting productivity of cofunctions with copatterns in Idris. However, due to various concerns discussed in Section \ref{sec:productivity}\todo{Remember this forward reference to Section \ref{sec:productivity}}, sized types may not be a desirable solution.

In relation to other methods for totality checking, Thibodeau\,\citep{Thibodeau11} presented an interesting comparison between sized types and termination checking by structural size change (such as the methods used for the size-change principle and the \texttt{foetus} termination checker). Here, Thibodeau concludes that sized types in general should be preferred over approaches examining structure, since they generally are more flexible. The main concern regarding sized types is that size annotations make the code harder to read, and seems unnecessary from the point of view of the user. Thibodeau expects, however, that it would be possible to reconstruct most size annotations automatically in a mature system. To our knowledge, it has yet to be investigated whether all size annotations could potentially be reconstructed, essentially making the user oblivious to the use of sized types.

\subsection{Methods Based On Time}
Atkey and McBride\,\citep{AtkeyMcBride13} have developed an experimental calculus with coinductive types in which productive programs can be typed by introducing ``clock variables'' to the type system. Much like the \texttt{Inf} operator described in Section~\ref{sec:stateinidris}, they use a \emph{guardedness type constructor} to indicate values that will only be available ``tomorrow''. These guarded types are then parameterized by clock variables, which has the effect that the output on a given ``day'' must be produced on that day, i.e. from the data that is available that day. Since Atkey and McBride quantify over clock variables in each definition, these provide only local constraints, expressing the relationship between input and output. This should be contrasted with the nature of sized types, where constraints can be expressed in terms of globally available types. In its current form, the system by Atkey and McBride is not an ideal solution for productivity checking definitions with copatterns, since it seems to require a certain style of programming in order to be usable. Furthermore, it still requires the user to specify ``time annotations'' for all definitions, which is not very different from sized types, although they only have local scope. A system based on local time constraints is quite interesting, however, and will be explored in greater detail as part of our proposal for a productivity algorithm in Section~\ref{sec:productivity}.

%Atkey and McBride\,\citep{AtkeyMcBride13} have developed an experimental calculus with coinductive types in which productive programs can be typed by introducing ``clock variables'' to the type system. The idea is to move the guardedness check from the syntactic level to the type level, by using \emph{guardedness type constructors} to represent values which can only be used in a guarded manner. These type constructors are then annotated by clock variables, such that it is possible for them to model guarded values which may only be available ``tomorrow''. The input and output of a function can then be typed such that they run on the same or related clocks, expressing certain guarantees about how many observations one can make on a given value. This idea of time will be revisited in Section \ref{sec:productivity}\todo{Remember this reference to Section \ref{sec:productivity}}. The difference between the approach taken by Atkey and McBride and full-blown sized types, is that they require fewer size annotation from the user, which is achieved by utilizing the described notion of time. The downside is that it does at this point require a certain style of programming which seems quite inflexible, making the approach less ideal for programming with copatterns.

Krishnaswami\,\citep{Krishnaswami13} devised a language for functional reactice programming based on time. In his system, it is impossible to write a program that references data which is not known to be well-typed at the current time. This is achieved by keeping track of a global clock, and then defining two operational semantics for evaluation of programs. One of these define evaluation of an expression at the current time, while the other has a ``tick relation'' which advances the global clock. Whenever the global clock is advanced, all values which can no longer be referenced due to time constraints are discarded from the environment. In effect, it becomes impossible to reference data which ``leak time'', i.e. should not currently be available according to the clock. The type system is extended with rules that make sure that no references to unavailable data can be made. This work is quite interesting in relation to the solution we propose in Section~\ref{sec:productivity}, since it also tries to discard unsafe data by using a notion of time. In contrast to Krishnaswami's approach which requires specific run-time behaviour in order to avoid ``space leaks'' (a specific type of memory leak, essentially), our proposal employs only static analysis. Nonetheless, the potential presence of memory leaks will have to be taken into account in our planned implementation of copatterns in Idris.


%!TEX root = ../main.tex
\section{Implementing Copatterns}
\label{sec:implementing-copatterns}
%!TEX root = ../main.tex
\section{Productivity of Cofunctions}
\label{sec:productivity}

Before delving into a description of our proposal for a productivity checking algorithm for definitions with copatterns, we will discuss the properties of other approaches, and why they may or may not be desirable.

\subsection{Lazy Evaluation: The Haskell Approach}
It may be compelling to think that we can simply model coinductive data in the same way as it is done in Haskell, by evaluating all expressions lazily. Then any productivity algorithm would work for Haskell as well, making the result more generally applicable. The lazy evaluation approach, however, is insufficient in several aspects.

First and foremost, if we insist on making no distiction between inductive and coinductive data in our type system (Haskell makes no distinction), subject reduction is lost in a dependently typed system such as Idris\,\citep{Abel13Copatterns}. While we will not unfold the argument in its entirety here, the problem lies in the fact that coinductive data is modeled as the \emph{construction} of infinite trees, making dependent pattern matching possible on codata, rather than in terms of its \emph{destructors}. Naturally, we want to preserve subject reduction in the Idris type system.

Secondly, Idris has the Church-Rosser property, meaning that distinct reduction strategies lead to the same normal form\,\citep{BradyIdrisImpl13}. Furthermore, the total part of Idris is strongly normalizing, such that it enjoys the \emph{strong} Church-Rosser property\,\citep{Turner04totalfunctional}. This means that not only will every reduction strategy leading to a normal form lead to the same normal form, but \emph{any} reduction strategy must lead to a normal form. In Haskell, many definitions will lead to a normal form under lazy evaluation, while eager evaluation would lead to infinite recursion. Therefore, exploiting lazy evaluation in the total part of Idris would mean that it would no longer have the strong Church-Rosser property.

\subsection{Productivity Checking with Sized Types}
As shown by Abel and Pientka\,\citep{Abel13Wellfounded}, verifying the productivity of coinductive definitions with copatterns is indeed possible using sized types. Thus, the main argument against using sized types is one of usability, from the point of view of the Idris user. Thibodeau\,\citep{Thibodeau11} emphasizes that size annotations generally make the code harder to read and seem unnecessary (see Section \ref{sec:related_work}). In our own experience, size annotations do place quite a burden of bookkeeping upon the user. While having more expressive types is generally considered a value tool for building correct programs, sized types do have a tendency to become something you use in order to satisfy the productivity checker, rather than because it leads to clearer type specifications. Because totality is optional in Idris (as opposed to in Agda, for instance), this means that in a practical programming setting, some users would most likely be inclined to switch off productivity checking for coinductive definitions, or refrain from using them entirely.

This argument is only valid as long as we have no way to fully reconstruct all size annotations. Once we do (if ever), the burden of sized types can be placed entirely upon the compiler. At such point, termination checking with sized types would seem like a fine approach.

\subsection{Guarded Corecursion}
In its original form presented by Coquand\,\citep{Coquand94} and Gim\'{e}nez\,\citep{Gimenez95}, the guardedness condition is generally too restrictive, as exemplified in Section \ref{sec:related_work}. The work done on the ESFP system by Telford and Turner\,\citep{Telford97ensuringstreams,Telford98ensuringthe} makes the guardedness criterion more generally applicable, even though some problems still remain, such as handling indirect application to corecursive functions. An example of this problem is the \texttt{g} function in Figure \ref{fig:TelfordTurnerProblems}, whose guardedness cannot be determined due to the indirect call to \texttt{id}. The function \texttt{f} is not considered guarded within their system because forward references may not be made to the rest of the process, i.e. the head of \texttt{f} cannot refer to the tail of \texttt{f}.

\begin{figure}
\begin{alltt}
f = (head (tail f)) :: (1 :: f)                 g = 1 :: ((fst funPair) g)
                                                where funPair = (id, id)
\end{alltt}
\caption{Two types of functions not deemed productive by the extended guardedness criterion by Telford and Turner\,\citep[Section 6.3]{Telford98ensuringthe}. Here, \texttt{::} is the cons operator for a hypothetical definition of streams without copatterns, and \texttt{id} is the identity function. The definitions of \texttt{head} and \texttt{tail} are as one would expect.} 
\label{fig:TelfordTurnerProblems}
\end{figure}

Abel\,\citep{Abel99terminationchecking} argues that termination analysis should abstract from the syntactic structure of the program being analyzed, since this makes the outcome of the analysis prone to small changes in program structure. While this argument seems sensible, the advantage of a syntactic check is that productivity does not depend on the user's ability or willingness to apply the right size annotations in the right places, as opposed to what is currently the case for sized types.

\paragraph{}
What we ultimately seek is a productivity checking algorithm for definitions with copatterns that covers as many cases as possible, without burdening the user too much. In the following, we will provide a detailed description of our proposal.
%An explanation of our proposal for productivity checking in collaboration with size-change termination.
% Why not guardedness?
% Why not sized types? 
% Why not the haskell way?
\subsection{Proposed Solution}
The base idea of our proposal is \textit{time measurement}. We wish to always know how many observations we can safely make on a given recursive reference. To do this we use time as an analogy. To define what happens at a given time we can only refer to what has happened earlier. Each observation in a copattern says what happens one time unit later. For this we will exemplify using days. If a given function \texttt{f} ``happens'' on day $F$, then one observation on \texttt{f} happens on day $F+1$, two observations on day $F+2$ etc. If we wish to define what happens on any given day $F+n$ we can only refer to things happening earlier, on day $D$ where $F+n\, \textgreater \,D$. Any definition that comply with this is said to be productive.

In Figure \ref{fig:zeros} we decorate a definition \texttt{zeros} with these time measures. Since both the \texttt{head} and the \texttt{tail} cases are one observation deep, they both get time measure $F+1$. Now we must look on the right hand sides of these observations. The \texttt{head} observation is simple as there are no recursive calls. This gets the time measure $F-1$ as we assume \texttt{Z} was known before time $F$. In the \texttt{tail} observation we have a recursive reference to \texttt{zeros}. This gets the time measure $F$ which is the time of \texttt{zeros}. We can say that \texttt{zeros} is productive because in the \texttt{head} case $F+1\,\textgreater\,F-1$ and in the \texttt{tail} $F+1\,\textgreater\,F$.

\begin{figure}
\begin{tabular}{l c}
\begin{minipage}{3in}
\begin{Verbatim}[commandchars=\\\{\},codes={\catcode`$=3\catcode`_=8}]
zeros : Stream Nat
head zeros = Z
tail zeros = zeros
\end{Verbatim}
\end{minipage} &
\begin{minipage}{3in}
\begin{Verbatim}[commandchars=\\\{\},codes={\catcode`$=3\catcode`_=8}]
zeros$_{F}$ : Stream Nat
head$_{F+1}$ zeros$_{F}$ = Z$_{F-1}$
tail$_{F+1}$ zeros$_{F}$ = zeros$_{F}$
\end{Verbatim}
\end{minipage}
\end{tabular}
\caption{An infinite stream of zeros. To the left is the implementation, and to the right is the implementation decorated with depth measures.}
\label{fig:zeros}
\end{figure}

In Figure \ref{fig:zerosprime} we see a faulty implementation of \texttt{zeros} called \texttt{zeros'}. This is not a productive implementation. We decorate the program in the same fashion as above, but this time the recursive reference in the \texttt{tail} case has time measure $F+1$ rather than $F$ because we make a right hand side observation on it. For this to be productive $F+1\,\textgreater\,F+1$ which it clearly is not. Therefore we can say that \texttt{zeros'} is not productive.

\begin{figure}
\begin{tabular}{l c}
\begin{minipage}{3in}
\begin{Verbatim}[commandchars=\\\{\},codes={\catcode`$=3\catcode`_=8}]
zeros' : Stream Nat
head zeros' = Z
tail zeros' = tail zeros'
\end{Verbatim}
\end{minipage} &
\begin{minipage}{3in}
\begin{Verbatim}[commandchars=\\\{\},codes={\catcode`$=3\catcode`_=8}]
zeros'$_{F}$ : Stream Nat
head$_{F+1}$ zeros'$_{F}$ = Z$_{F-1}$
tail$_{F+1}$ zeros'$_{F}$ = tail$_{F+1}$ zeros'$_{F}$
\end{Verbatim}
\end{minipage}
\end{tabular}
\caption{A non-productive implementation of zeros.}
\label{fig:zerosprime}
\end{figure}

\subsubsection{``Higher-order'' Cofunctions}\todo{Need better name for this kind of function}

Sometimes we want to write definitions that depend on other functions with coinductive parameters. When decorating \texttt{nats} in Figure \ref{fig:nats} we can not give the right hand side of the \texttt{tail} observation an accurate time measure as we don't know what \texttt{map} does to \texttt{nats}. We must first analyse \texttt{map} to see the effect it has on it parameters.

\begin{figure}
\begin{Verbatim}[commandchars=\\\{\},codes={\catcode`$=3\catcode`_=8}]
nats$_{N}$ : Stream Nat
head$_{N+1}$ zeros$_{N}$ = Z$_{N-1}$
tail$_{N+1}$ zeros$_{N}$ = map S nats$_{?}$
\end{Verbatim}
\caption{A Stream of all the natural numbers.}
\label{fig:nats}
\end{figure}

When analysing \texttt{map} we introduce a new time parameter for each coinductive input the function takes. This is not used for checking \texttt{map} but is used for callers of map to see how many observation \texttt{map} performs on a given input. This means that instead of just checking the productivity of \texttt{map} we also create a \textit{specification} for \texttt{map} that other functions can then use. The caller can then judge if the call would be productive or not.

\begin{figure}
\begin{Verbatim}[commandchars=\\\{\},codes={\catcode`$=3\catcode`_=8}]
map$_{M, S}$ : (a -> b) -> Stream a -> Stream b
head$_{M+1}$ (map f s)$_{M}$ = f (head$_{S+1}$ s$_{S}$)
tail$_{M+1}$ (map f s)$_{M}$ = map f (tail$_{S+1}$ s$_{S}$)
\end{Verbatim}
\caption{The map function for Streams.}
\label{fig:map}
\end{figure}

Making the specification is fairly simple. We just have to find the worst case for every input. In the case of \texttt{map} the worst case for time parameter $S$ is $S+1$. This tells us that \texttt{map} performs at most one observation on the input. This, however, does not complete the specification. Since we are interested in knowing how many observations are safe to do, and \texttt{map} only observes on its input under its own observations, this allows us to do another observation.\todo{Does this need to be more explicit?} Using the day analogy again this roughly means that tomorrow \texttt{map} will perform the observations of tomorrow. This gives us $S+1-1$ or just $S$, which means that the specification for \texttt{map} does not perform any additional observations on its input. This means we can now complete the annotation of \texttt{nats} in Figure \ref{fig:natsComplete}. Since the specification of \texttt{map} tells us that the input remains unchanged by \texttt{map} we know that the time measure of the call to \texttt{map} with \texttt{nats} as input is $N$. We can therefore conclude that \texttt{nats} is productive as $N+1\,\textgreater\,N$.

\begin{figure}
\begin{Verbatim}[commandchars=\\\{\},codes={\catcode`$=3\catcode`_=8}]
nats$_{N}$ : Stream Nat
head$_{N+1}$ zeros$_{N}$ = Z$_{N-1}$
tail$_{N+1}$ zeros$_{N}$ = map S nats$_{N}$
\end{Verbatim}
\caption{A Stream of all the natural numbers with full time annotations.}
\label{fig:natsComplete}
\end{figure}

\subsubsection{Mutual Recursion}
For checking productivity of mutual recursive coinductive definitions we assume that we know what definitions are mutually recursive. Similar to the previous example we then make a specification for the function that can then be used to check the productivity of the callers. Instead of this specification saying what happens to an input parameter it says what happens to a specific call. If we for example have two mutually recursive definitions \texttt{f} and \texttt{g}. For \texttt{f} we define specification $F_{g}$ saying what happens to the time parameter $F$ in \texttt{g} and vice versa we define $G_{f}$ for \texttt{g}. For the \texttt{tail} case in \texttt{f} to be productive we need that $F+1\,\textgreater\,F_{g}$ and for \texttt{g} we need $G+1\,\textgreater\,G_{f}+1$ to hold.

\begin{figure}
\begin{Verbatim}[commandchars=\\\{\},codes={\catcode`$=3\catcode`_=8}]
f$_{F}$ : Stream Nat
head$_{F+1}$ f$_{F}$ = Z$_{F-1}$
tail$_{F+1}$ f$_{F}$ = g$_{F_{g}}$

g$_{G}$ : Stream Nat
head$_{G+1}$ g$_{G}$ = Z$_{G-1}$
tail$_{G+1}$ g$_{G}$ = tail$_{G_{f}+1}$ f$_{G_{f}}$
\end{Verbatim}
\caption{A simple mutual recursive coinductive definitions.}
\label{fig:mutRec1}
\end{figure}

Finding these specifications again is simple. To find $G_{f}$ we look at the references to \texttt{g} in \texttt{f}. We see that there is one observation on the left hand side and none on the right hand side. For the same reason as with \texttt{map} we subtract one for the left hand ones and add one for the right hand side ones. This means that $G_{f} = G - 1$. Using the same approach for finding $F_{g}$ we get that $F_{g} = F$. We can now substitute these values into the inequalities from earlier and get that $F+1\,\textgreater\,F$ and $G+1\,\textgreater\,G - 1$. Therefore \texttt{f} and \texttt{g} are productive.

\subsubsection{Limitations}
There are a few limitations that we have discovered (and likely more that we have not). In this section we will present the known limitations and discuss how to possibly solve them.

One limitation is if an observation refers to something that is trivially productive, but under more observations. Consider the example in Figure \ref{fig:forwardRef}. This definition is productive, but will be discarded by our approach as it does not hold that $H+1\,\textless\,H+2$. This can be solved by checking such references directly. In this case since \texttt{head tail f} is directly defined we can say that \texttt{head f} is productive if \texttt{head tail f} is productive.

\begin{figure}
\begin{Verbatim}[commandchars=\\\{\},codes={\catcode`$=3\catcode`_=8}]
h$_{H}$ : Stream Nat
head$_{H+1}$ h$_{H}$         \,= head$_{H+2}$ tail$_{H+1}$ h$_{H}$
head$_{H+2}$ tail$_{H+1}$ h$_{H}$ = Z$_{H-1}$
tail$_{H+2}$ tail$_{H+1}$ h$_{H}$ = tail$_{H+1}$ h$_{H}$
\end{Verbatim}
\caption{An example of forward referencing.}
\label{fig:forwardRef}
\end{figure}

Another limitation is functions returning arbitrary functions. In Figure \ref{fig:funList} we cannot give an accurate time measure to \texttt{(funList !! 1) i} because we do not know anything about what \texttt{funList !! 1} does to \texttt{i}. A way of attempting to solve this could be to try and estimate a worst case scenario for \texttt{funList}. This, however, involves a lot of unfolding risking making the productivity checker very slow, while still not being a very accurate analysis. We have not been able to find a desirable solution to this problem.

\begin{figure}
\begin{Verbatim}[commandchars=\\\{\},codes={\catcode`$=3\catcode`_=8}]
i$_{I}$ : Stream Nat
head$_{I+1}$ i$_{I}$ = Z$_{I-1}$
tail$_{I+1}$ i$_{I}$ = (funList !! 1) i$_{?}$

funList : [Stream a -> Stream a]
funList = [id, id, id]
\end{Verbatim}
\caption{}
\label{fig:funList}
\end{figure}
%!TEX root = ../main.tex
\section{Copatterns in Idris}
\label{sec:copatterns_in_idris}
%!TEX root = ../main.tex
\section{Future Work}
\label{sec:future_work}
The above sections give rise to several directions for future work. Specifically, we will have to:
\begin{itemize}
\item Formalize the productivity algorithm presented in Section~\ref{sec:productivity}.
\item Show the correctness of the productivity algorithm.
\item Investigate further the internals of the Idris implementation.
\item Find out how codata definitions by observations and corecursive functions with copatterns can be elaborated to \TT.
\end{itemize}
Although the productivity algorithm was described in natural language in Section~\ref{sec:productivity}, a more operational formalization of the algorithm is needed if we are to implement it. A possible step on the way to implementing the algorithm in Idris would be to extend \textit{findus} with dependent types in order to get a realistic testing environment, and then incorporate the productivity algorithm for copatterns into this system. Then, once the algorithm works in this setting, it should be easier to realize in Idris.

To ensure that our productivity algorithm is usable, we will need to prove its correctness. This could possibly be done by reducing the system of time measurement to a subset of sized types.

Since the constructs described in Section~\ref{sec:copatterns_in_idris} must be added to Idris, we will need to delve deeper into the Idris implementation. In particular, we need to investigate how to elaborate these constructs to the core type theory. Additionally, the productivity algorithm must be fit into the system alongside the size-change termination checker.

%!TEX root = ../main.tex
\section{Conclusion}
\label{sec:conclusion}
Throughout this project, our main objective has been to prepare ourselves for for our coming master's thesis, where we plan to implement coinductive data by observations and corecursive function definitions with copatterns in Idris. We have sought to achieve this by first studying the theory behind coinductive data and copatterns. In this process, we found that coinductive data is dual to inductive data, in the sense that where the latter is defined in terms of constructors, the former is most naturally defined in terms of the observations we can make on it. Copatterns provide an elegant way of defining corecursive functions by observations, and thus seem like a valuable addition to Idris. 

To understand the implementation-specific details behind copatterns, we subsequently went ahead and implemented them in a simple functional programming language. From this, we found that neither coinductive types nor copatterns proved to be complex to implement. Coinductive types can be modeled as recursive types, while copatterns be added as a language construct similar to pattern matching. Therefore, we expect that the greatest challenge ahead will not be the implementation of copatterns in itself, but rather how they can be added to the existing Idris system.

Following an investigation into the existing literature on the size-change principle, we concluded that no implementation of the principle exists which would be suitable as a productivity checking algorithm for definitions with copatterns. Of the other approaches considered, sized types seems most promising. We decided against a recommendation of sized types for ensuring productivity, however, since it would require the user to provide too many annotations. The solution proposed in Section~\ref{sec:productivity} is thus entirely a syntactic check. The basic idea of the algorithm is that the outcome of an observation at any time can only be defined in terms of observations that are already known at that time. Non-productive definitions are therefore exactly those definitions which require knowledge about ``the future''.

Concerning Idris, we expect that it will not be necessary to make any changes to the core type theory, since coinductive data defined by observations and function definitions with copatterns can be reduced to already existing constructs. The interplay between the size-change termination checker and our proposed productivity checking algorithm will be interesting, since it is possible for definitions with copatterns to be size-change terminating without being productive according to our algorithm.





\bibliography{bibliography}
\end{document}