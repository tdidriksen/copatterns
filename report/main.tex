 %!TEX TS-options = -shell-escape
\documentclass[oribibl]{llncs}
\pagestyle{headings}
\usepackage{natbib}
\bibliographystyle{alpha}
\newenvironment{changemargin}[2]{%
\begin{list}{}{%
\setlength{\topsep}{0pt}%
\setlength{\leftmargin}{#1}%
\setlength{\rightmargin}{#2}%
\setlength{\listparindent}{\parindent}%
\setlength{\itemindent}{\parindent}%
\setlength{\parsep}{\parskip}%
}%
\item[]}{\end{list}}

\usepackage[utf8]{inputenc}
\usepackage{alltt}
\usepackage{url}
\usepackage{todonotes}
\usepackage{fixltx2e} % for subscript
\usepackage{fancyvrb}
\usepackage{amsmath}

\newcommand{\Idris}{\textsc{Idris}}
\newcommand{\IdrisM}{\textsc{Idris}$^-$}
\newcommand{\TT}{\textsf{TT}}

\begin{document}
\mainmatter
\title{Understanding and Implementing Copatterns}
\author{Sune Alk\ae{}rsig and Thomas Hallier Didriksen \\
\email{\{sual, thdi\}@itu.dk}}

\institute{IT University of Copenhagen, Rued Langgaards Vej 7, 2300 Copenhagen S, Denmark}
\maketitle

\begin{abstract}
While inductive data can be understood in terms of constructors or introduction rules, coinductive data can be understood in terms of observations or elimination rules. We explore the idea of programming by observations, where coinductive definitions are defined with \emph{copatterns}, a construct for defining the results of observations. Through the implementation of a small functional language with coinductive types and copatterns, we investigate the underlying properties of this construct. Furthermore, after providing a survey of existing approaches to productivity checking, we devise a productivity checking algorithm for definitions with copatterns which is based on a purely syntactic check. These efforts serve as preparatory work for our coming master's thesis, where we plan to implement copatterns in Idris. Therefore, an analysis of the needed changes to the Idris compiler is also provided.
\keywords{Copatterns, Coinductive data, Corecursion, Idris, Totality}
\end{abstract}

%!TEX root = ../main.tex
\section{Introduction}
\label{sec:introduction}
%Copatterns. This is an preparatory study for our master's thesis.
%How do we fit copatterns into Idris?
In the world of functional programming, the use of inductive data types has long been a standard practice. Inductive data types enable users to define fundamental data structures such as lists, trees, and graphs, and to analyze data by pattern matching. A fundamental property of inductively defined data is that it is inherently finite. Since it is defined in terms of constructors, infinite structures cannot be defined inductively in systems with finite memory unless certain evaluation strategies are abandoned. In particular, programming languages such as Haskell have successfully been able to blur the lines between finite and infinite structures through the power of lazy evaluation. An alternative approach is to define infinite data explicitly by the use of \emph{co}inductive data types. These were first implemented by Hagino for his SymML language\,\citep{Hagino89}, allowing users to define coinductive data in terms of its \emph{destructors}. This means that it does not make sense to think of coinductive data as something that is constructed. Rather, coinductive data can be thought of as something that can be taken apart and observed, and therefore the terms ``observation'' and ``destructor'' are used interchangably. Explicit coinductive data types have since been added to several other programming systems, examples of which are the Coq proof system\,\citep{Coq:manual}, Agda\,\citep{Norell:thesis}, Beluga\,\citep{Pientka10}, and Idris\,\citep{BradyIdrisImpl13}.

While most (if not all) systems with inductive data types have a pattern matching construct for analyzing data, many systems with coinductive data types do not have dedicated constructs for manipulating destructors. Instead, coinductive data is also subjected to pattern matching. In SymML, Hagino describes a \texttt{merge} construct for describing how a coinductive definition can be destructed, an example of which is shown in Figure~\ref{fig:merge_SymML}.

\begin{figure}
\begin{alltt}
codatatype 'a inflist = head is 'a & tail is 'a inflist;

fun iseq1() = merge head <= 1
              & tail <= iseq1();
\end{alltt}
\caption{An example of the \texttt{merge} construct in SymML. Here, \texttt{inflist} is a definition of an infinite list, and \texttt{iseq1} is an infinite list of ones.}
\label{fig:merge_SymML}
\end{figure}

In Coq, no such construct exists, and the same is the case for Idris. In the wake of work done by Abel and Pientka\,\citep{Abel13Wellfounded} and Abel, Pientka, Thibodeau, and Setzer\,\citep{Abel13Copatterns}, a construct similar to Hagino's \texttt{merge} has recently been implemented in Agda. This construct is called a \emph{copattern}. By allowing projections on the left-hand side of function definitions, copatterns enable users to define corecursive functions in terms of observations. In effect, coinductive data no longer has to be analyzed with pattern matching, but can more naturally be expressed by the observations that can be made on it. 

\subsection{Goals and Contributions}
This project serves as a preparation for our master's thesis, where we plan to implement copatterns and coinductive data defined by observations in Idris. The goals of this project are:

\begin{itemize}
\item To understand the ideas behind copatterns.
\item To understand how copatterns are used in practice, for example by studying Agda.
\item To understand how copatterns can be implemented.
\item To implement a simple functional programming language with copatterns.
\item To understand how the current termination checker in Idris works by implementing a similar termination checker employing the size-change principle\,\citep{LeeJones01SizeChange} for said language.
\end{itemize}

As a result of pursuing these goals, our contributions are:

\begin{itemize}
\item An investigation into whether the intended changes can be implemented in Idris without changing its core type theory.
\item A survey of the literature describing different implementations of the size-change principle. This is done in order to find out whether it is feasible to extend the size-change termination checker in Idris to work as a productivity checker for corecursive function definitions with copatterns. Additionally, existing approaches to productivity checking will be explored, for example the guardedness principle.
\item A proposal for a productivity checking algorithm for corecursive function definitions with copatterns.
\end{itemize} 

\subsection{Outline}
In Section~\ref{sec:background}, we describe the core concepts behind coinductive data, copatterns, and productivity. Section~\ref{sec:related_work} provides a survey of the relevant literature, in particular with respect to termination and productivity checking. The implementation of a small functional language with copatterns is described in Section~\ref{sec:implementing-copatterns}. In Section~\ref{sec:productivity}, we present our proposal for a productivity algorithm for corecursive functions with copatterns, while also discussing the virtues of other approaches to productivity checking. Section~\ref{sec:copatterns_in_idris} explains the necessary changes in order for us to add definitions with copatterns to Idris, and finally, Section~\ref{sec:conclusion} provides a conclusion for the entire report.

%!TEX root = ../main.tex
\section{Background}
\label{sec:background}

\subsection{Why Do We Care About Totality?}
In a conventional programming setting, we know that our program results in a value \emph{if} it terminates \emph{and} does not result in a run-time error, both of which are possible. We have no guarantees about the behaviour of a program, aside from the confidence we have in ourselves as developers. Even if we have proof that our program acts correctly according to its specification, we are only guaranteed that our program is \emph{partially} correct, unless we prove that it is also total.

Totality covers two terms: \emph{termination} and \emph{productivity}. A function on inductive data terminates whenever it has consumed all of its input. A function on coinductive data is productive if it can always produce a finite prefix of its result in finite time. Both terms imply that any invocation of a total function cannot result in an infinite loop, but they do so in different ways. 

Termination talks about input. Because we know that inductive data is always finite, it is reasonable to expect a terminating function to be able to consume all of its input. When there is no more input, the function should result in a value, which in turn implies that a terminating function must be defined for all cases of its input. 

Productivity talks about output. Since coinductive data is possibly infinite, we cannot rely on the function consuming all of its input. Instead, we require that it must always produce a result in finite time. While this might sound vague, it implies that even though the function produces coinductive data, it must produce it in infinitely many (finite) chunks, which can be returned one at a time.

Making programs total have multiple benefits. First of all, it gives us a guarantee that an output is always produced, which means that we can substitute a degree of confidence with a guarantee. Secondly, due to the Curry-Howard correspondence we can use total programs as proofs, which enables us to establish strong guarantees about their correctness. In Idris, partial functions are not evaluated by the type checker (the type checker might loop forever), and thus can never consitute part of a proof. A common argument against total programming is that we cannot create programs which are not meant to terminate, such as servers and operating systems. Whoever makes such an argument has, regrettably, mistaken \emph{totality} for \emph{termination}. It is true that servers and operating systems are not meant to terminate, but they can be productive, a notion which is well-suited for systems that need to be responsive.

In short, we care about totality because it provides us with guarantees, and ultimately enables us construct provably correct programs.

% Non-termination is a possibility
% Not used in proof --- not expanded by type checker
% We can write total servers and operating systems!
% termination vs. productivity

\subsection{Codata}
What is codata?

\subsection{The Current Status of Codata in Idris}
Coinductive data types already exist in Idris, although they are not defined by observations\todo{Make sure that definitions by observations are mentioned before this point}. Instead, they are defined by constructors, as exemplified in Figure~\ref{fig:stream_current}. 

\begin{figure}
\begin{alltt}
codata Stream : Type -> Type where
  (::) : a -> Stream a -> Stream a
\end{alltt}
\caption{A stream definition as it currently looks in Idris.}
\label{fig:stream_current}
\end{figure}

At first glance, the type of the \texttt{(::)} constructor looks reasonable. Taking this type at face value, let us try to imagine how a \texttt{Stream} of all the natural numbers could be built:

\begin{alltt}
(0 :: (1 :: (2 :: (3 :: (4 :: (5 :: \ldots))))))
\end{alltt}

This seems to be impossible: Building a \texttt{Stream} requires a new \texttt{Stream}, which again requires a new \texttt{Stream}, leading to an infinite chain of \texttt{Stream}s. Is \texttt{Stream} inhabited at all? Fortunately, the answer is yes, the reason being that the syntax hides an important detail. Internally, the \texttt{(::)} constructor only requires a lazily evaluated \texttt{Stream} as its second argument. Instead of giving an actual stream, we can specify a promise that the rest of the \texttt{Stream} can be generated later. This is captured by the \texttt{Inf} type operator in Figure \ref{fig:stream_current_Inf_and_nats}. The recursive call in \texttt{nats} will be lazily evaluated, and therefore not cause an infinite computation.

\begin{figure}
\begin{alltt}
codata Stream : Type -> Type where
  (::) : a -> Inf (Stream a) -> Stream a

nats : Nat -> Stream Nat
nats n = n :: nats (S n)
\end{alltt}
\caption{A stream definition as it currently looks in Idris with the implicit \texttt{Inf} operator made explicit. This allows us to define an infinite sequence of natural numbers as shown with \texttt{nats}.}
\label{fig:stream_current_Inf_and_nats}
\end{figure}

This scheme of lazily evaluating coinductive arguments is applied automatically to any constructor in a \texttt{codata} declaration during elaboration, and \texttt{Inf} is therefore omitted in the concrete Idris syntax.

To ensure the productivity of functions returning codata, Idris has a productivity checker which analyzes a program according to the \emph{guardedness} principle\,\citep{Coquand94,Gimenez95}: Any recursive call must appear directly under a constructor. In the \texttt{nats} example, the recursive call appeared directly under the \texttt{(::)} constructor. If we try to define the same function without the \texttt{Nat} argument (shown in Figure \ref{fig:nats'}), however, the productivity checker rejects the program, complaining that \texttt{nats'} is possibly not total. In this case the recursive call does not appear directly under \texttt{(::)}, but is wrapped in a call to \texttt{map}. Therefore, even though \texttt{nats'} is in fact productive, it is not productive according to the guardedness principle.

\begin{figure}
\begin{alltt}
nats' : Stream Nat
nats' = Z :: map S nats'
\end{alltt}
\caption{A stream of all the natural numbers which is productive, but not according to the guardedness principle.}
\label{fig:nats'}
\end{figure}

In its original formulation (which is also the one implemented for Idris), the guardedness principle is quite simple, but also rather conservative. The productivity checking algorithm we present in Section~\ref{sec:productivity} accepts more functions as being productive, but is arguably not as simple.

\subsection{Copatterns}
What are copatterns? (conceptually)
% Copatterns in Agda

% Inductively defined data can be analyzed by defining functions on that data by traditional pattern matching, where a \emph{pattern} is a (finite and valid) combination of constructor applications for the given data. In this respect, 

% \begin{figure}
% \begin{alltt}
% pow2 : Stream Nat
% head pow2 = S Z
% head (tail pow2) = S (S Z)
% tail (tail pow2) = zipWith _+_ (tail pow2) (tail pow2)
% \end{alltt}
% \caption{A definition of an infinite stream of powers of 2 using copatterns.}
% \end{figure}

% by defining functions on that data using traditional pattern 

% Inductively defined data can be analyzed by defining functions in terms of a set of \textit{patterns}, where a pattern constitutes a recognizable structure in the data we are analyzing. 




% Because we know that the data has been constructed using a finite number of data constructors, the number of ways we can take that same data apart is also finite by definition. 


% Consider a standard list data structure, defined in Haskell-like syntax:

% \begin{alltt}
% data List a = Nil 
%             | Cons a (List a)
% \end{alltt}

% \texttt{List a} consists of two data constructors, \texttt{Nil} and \texttt{Cons}. Because these two constructors constitute the only that we can construct a \texttt{List a}, 

%!TEX root = ../main.tex
\section{Related Work}
\label{sec:related_work}
This section gives a survey of relevant literature, mostly covering approaches to ensuring termination of programs defined over both inductive and coinductive data.

\subsection{Guarded Corecursion}
The basic idea of guarded corecursion is that the productivity of cofunctions can be ensured by a purely syntactic check. It was first proposed by Coquand\,\citep{Coquand94} as an important part of a ``guarded proof induction principle'' for a proof system containing coinductive definitions, inspired by a similar approach in the area of process calculi by Milner\,\citep{Milner82}. In continuation of Coquand's efforts, Gim\'{e}nez applied the method to the Calculus of Constructions in order to avoid the introduction of non-normalizable terms\,\citep{Gimenez95}. 

% ESFP
\citep{Telford97ensuringstreams}
\citep{Telford:jucs_6_4:ensuring_termination_in_esfp} 




\subsection{Sized Types}

\subsection{Size-change Termination}
The size-change principle for termination was first proposed for a strict first-order functional language (without loop constructs) by Lee, Jones, and Ben-Amram\,\citep{LeeJones01SizeChange}. The principle essentially states that if infinitely many recursive calls to a function would lead to infinite decrease in some parameter value, then the function must be terminating, since any value of an inductive type must have a finite structure. This last condition is of particular importance, as the size-change principle cannot in general decide termination of functions whose parameters do not have a well-founded order. In the graph formulation of their algorithm, termination is decided by identifying any recursive calls (both direct and indirect) through call graph completion, and then constructing a ``size-change graph'' for each call. These are then used to determine whether infinite descent in some parameter value is present.

Since its first-order formulation, the principle has been proven to work for more expressive cases. Jones and Bohr\,\citep{Jones04Untyped} showed that size-change termination can be applied to the untyped lambda calculus using abstract interpretation. Interesting results of this work include size-change termination of programs employing the Y-combinator.

Following the work on the untyped lambda calculus, Sereni and Jones generalized the size-change principle to handle a higher-order functional language with user-defined data types and general recursion\,\citep{Sereni05terminationanalysis,Sereni06Phd}. Here, a termination criterion is presented which works for arbitrary control-flow graphs, and in turn is able to decide termination of lazy functional programs. A key point in this work is how different approaches to control-flow and call graph construction may influence the preciseness of the termination analysis.

Most formulations of the size-change principle only work on data for which some well-founded order exists. Nevertheless, Avery\,\citep{Avery06} presented a formulation in which it is possible to determine size-change termination for non-well-founded data types --- in particular, this formulation is shown to work for a language with an integer type. Instead of identifying infinite descent using a well-founded partial order on values, the analysis is based on a decrease in invariants which are found to hold for each program point. Thus, if the value of some invariant (which can involve arbitrarily many values) can be shown to decrease on every passage of a program point, then the program terminates. While an approach to size-change termination involving non-well-founded data is definitely closer to a usable definition for deciding the productivity of cofunctions with copatterns, Avery's formulation is insufficient or impractical in several aspects. First, it is still assumed that termination is determined solely on the basis of an infinitely decreasing property. However, many cofunctions defined with copatterns exhibit no such property; Indeed, one of the virtues of cofunctions is exactly the ability to define infinite structures. Secondly, the formulation works on a graph of program points. Since it is valid for a cofunction to be defined in terms of itself, non-termination for copatterns only becomes apparent when observations are defined in terms of themselves (and as such, they can be said not to be defined at all, see Section \ref{sec:background}\todo{remember this backwards reference to Section \ref{sec:background}}). Thus, constructing a graph of observations would be necessary for this approach to work, and such an undertaking might not be ideal (see Section \ref{sec:productivity}\todo{remember this forward reference to Section \ref{sec:productivity}})


The ideas presented in this paper are fundamental to our understanding of the totality checker implemented for Idris. 
%!TEX root = ../main.tex
\section{Implementing Copatterns}
\label{sec:implementing-copatterns}
%Description of our language.
%Explains the interesting details of implementing copatterns.
To explore the implementation of copatterns we have made a small toy language called \textit{findus} implemented in Haskell. It is a functional programming language with first-order functions and simple first-order types. It has the following features:

\begin{itemize}
\item Inductive Types
\item Coinductive Types
\item Let bindings
\item Pattern Matching 
\item Copatterns
\end{itemize}

Inductive and coinductive types are implemented as iso-recursive types\,\cite[Section~20.2]{Pierce:2002:TPL:509043}, or $\mu$X.T and $\nu$X.T respectively. This means that their implementations are very similar. Inductive types (\texttt{TRecInd} in Figure~\ref{fig:types}) are defined by the name of the type and then a variant type (\texttt{TVari}). The variant type consists of a list which defines the constructors, where the string part is the name of the constructors and the list of types are the types of the constructor parameters. The coinductive type (\texttt{TRecCoind}) is defined by its name and a list of observations that is similar to the variant type definition, except that there is only one type per observation. The coinductive argument to an observation (the object we observe) can be inferred, so only its return type is specified. A possible extension could be to allow observations to take parameters, but we decided against this since the focus of this project is copatterns, and matching on these parameters would just be standard pattern matching. Defining coinductive data types is done with the \texttt{DCodata} definition in Figure~\ref{fig:expr}, which is defined by the name of the type and the coinductive type that defines it.

\begin{figure}
\begin{Verbatim}[commandchars=\\\{\},codes={\catcode`$=3\catcode`_=8}]
data Type =
  \vdots
  | TVari [(String, [Type])]          // Sum of Products (Variant Type)
  | TRecInd String Type               // $\mu$X.T
  | TRecCoind String [(String, Type)] // $\nu$X.T
  | TRecTypeVar String                // Recursive type variable
  | TGlobTypeVar String               // Global type variable
\end{Verbatim}
\caption{Part of the \texttt{Type} type in findus}
\label{fig:types}
\end{figure}

Dual to pattern matching we need a way to make observations on our coinductive types. For this we have the \texttt{EObserve} expression in Figure~\ref{fig:expr}. Such an expression is defined by the coinductive type that is being observed and a list of observations. These observations consist of the name of the observation followed by an expression defining that observation. This \texttt{EObserve} expression models copatterns.

\begin{figure}
\begin{alltt}
type Params = [(String, Type)]

data Expr =
  \vdots
  | EVar String                              // Variable
  | EObserve Type [(String, Expr)]           // Observe
  \vdots

data Defi =
  \vdots
  | DData String Type                        // Data
  | DCodata String Type                      // Codata
  | DGlobLet String Type (Maybe Params) Expr // Global Let
  \vdots
\end{alltt}
\caption{Part of the \texttt{Expr} and \text{Defi} type in findus}
\label{fig:expr}
\end{figure}

\begin{figure}
\begin{alltt}
-- Natural numbers

natBody :: Type                         -- nat constructors
natBody = TVari [
            ("Z", []),                  -- Z Constructor
            ("S", [TRecTypeVar "nat"])  -- S Constructor
          ]

natRec :: Type
natRec = TRecInd "nat" natBody          -- nat inductive type

nat :: Defi
nat = DData "nat" natRec                -- nat data definition

-- Nat Stream

natStreamBody :: [(String, Type)]          -- natStream observations
natStreamBody = [
                  ("head", TGlobTypeVar "nat"),     -- Head observation
                  ("tail", TRecTypeVar "natStream") -- Tail observation
                ]

natStreamRec :: Type                    -- natStream coinductive type
natStreamRec = TRecCoind "natStream" natStreamBody  

natStream :: Defi                       -- natStream codata definition
natStream = DCodata "natStream" natStreamRec  
\end{alltt}
\caption{Natural numbers and a stream of natural numbers}
\label{fig:natandstream}
\end{figure}

Now that we have a way of representing inductive and coinductive types, let us look at some examples of their use. In Figure~\ref{fig:natandstream}, the abstract syntax representation of natural numbers (\texttt{nat}) and a stream of natural numbers (\texttt{natStream}) is shown. The type \texttt{TGlobTypeVar} is used for referencing globally defined types, and \texttt{TRecTypeVar} is a reference to the type currently being defined. From these we can define the stream of zeros in Figure~\ref{fig:astzeros}. We wrap \texttt{zerosExpr} in a \texttt{DGlobLet} which is a top-level let binding existing in global scope. An \texttt{DGlobLet} is defined by its name, its annotated type, its parameters (if any), and its body.

\begin{figure}
\begin{alltt}
zerosLet :: Expr
zerosLet = DGlobLet "zeros" (TGlobTypeVar "natStream") Nothing zerosExpr

zerosExpr :: Expr
zerosExpr = EObserve (TGlobalTypeVar "natStream") [
            ("head", EVar "Z"),
            ("tail", EVar "zeros")
        ]
\end{alltt}
\caption{A stream of zeros.}
\label{fig:astzeros}
\end{figure}

\subsection{Syntax}
We have implemented a simple concrete syntax which requires a brief explain. The language uses \texttt{case} expressions for inductive pattern matching. For copatterns, we use an \texttt{observe as} construct inspired by the case expression. The idea is to describe the different observations of a given coinductive type. Examples of the corresponding concrete syntax to the abstract syntax from above can be seen in Figure \ref{fig:concrete_syntax}.

\begin{figure}
\begin{alltt}
data nat = \{Z | S nat\}

codata natStream = \{head nat | tail natStream\}

let zeros : natStream =
  observe natStream as
    head -> Z ;
    tail -> zeros
\end{alltt}
\caption{Concrete \textit{findus} syntax for \texttt{nat}, \texttt{natStream}, and \texttt{zeros}.}
\label{fig:concrete_syntax}
\end{figure}

\subsection{Type Checking}
\textit{findus} is type checked according to the rules described by Pierce\,\cite{Pierce:2002:TPL:509043}. This means that all types are annotated, and none are inferred. Pierce does, however, not give typing rules for codata and copatterns.

When type checking copatterns, we know from Section~\ref{sec:copatterns} that the type of the outcome of an observation depends on the observation made. This means that the expression defining the outcome of an observation must be the same as the type of the observation. In the example of \texttt{natStream}, the expression defining a \texttt{head} observation must be of type \texttt{TGlobTypeVar "nat"}. As such, type checking copatterns is fairly simple, as we just need to check that the types of the outcome and the definition of an observation match.

\subsection{Evaluation}
Due to the nature of coinductive types, they are not actually evaluated until they are observed. The result of evaluating an observation might be a new coinductive definition that we again cannot evaluate before it is observed. There are situations, however, where coinductive definitions can be rewritten to something else. Consider the example of \texttt{nats} in Figure~\ref{fig:nats_copatterns}, which is defined with copatterns. Evaluating \texttt{nats} in itself does not make much sense as there is nothing to evaluate; evaluation does not occur until we observe \texttt{nats}. When evaluating \texttt{nats}, we simply examine at the definitions and unfold them. Assuming an evaluation function \texttt{eval}, we have the following examples of reductions:

\begin{Verbatim}[commandchars=\\\{\},codes={\catcode`$=3\catcode`_=8}]
eval (head nats) = Z
eval (tail nats) = map S nats
eval (head tail nats) = eval (head (map S nats)) 
                      = eval (S (head nats)) 
                      = S Z
eval (tail tail nats) = eval (tail (map S nats)) 
                      = map S (tail nats)
\end{Verbatim}

We evaluate by unfolding until we either cannot unfold a coinductive definition anymore, or get an expression that is not coinductive. A coinductive definition cannot be unfolded if additional observations are required in order to reduce it. For example, \texttt{nats} requires additional observations, but \texttt{tail nats} is directly defined as \texttt{map S nats}. Unfolding definitions in this manner could lead to an infinite series of unfolds. If we want a guarantee that this will not happen in a given situation, our coinductive definitions must be checked for productivity. Productivity will be discussed further in Section~\ref{sec:productivity}. 

\subsection{Termination Checking}
In order to get an in-depth understanding of the size-change principle (as explained in Section~\ref{sec:related_work}), we have implemented a termination checker for \textit{findus} based on the original formulation of the principle\,\citep{LeeJones01SizeChange}. For the size-change principle, two abstract notions are used: \emph{control points} and \emph{data positions}\,\citep{Krauss07certifiedsizechange}. In short, control points are the entities which must be terminating, while data positions are the entities that change size. For our implementation, the control points are \textit{findus} functions, and the data positions are the parameters. Inspired by Hyvernat\,\citep{Hyvernat13}, sizes are modeled with an inductive data type. Since we do not have any primitive data types, values can only become structurally smaller through pattern matching, and structurally larger by constructor application.

Termination is detected by first constructing a call graph of all direct calls from one function to another. For each of these calls, a size-change graph is constructed. A size-change graph is a 3-tuple \texttt{(f, arcs, g)}, where \texttt{f} and \texttt{g} are functions, and \texttt{arcs} is a set of \emph{size-change arcs}. Size-change arcs capture the relationship between a parameter and the argument given at a function call. This relationship is either \emph{descending} or \emph{non-increasing}, depending on the changes in size. If the change in size is increasing, there is no relationship. When the size-change graphs have been constructed, indirect recursive calls are found by detecting cycles in the call graph by simple graph traversal. For each of these cycles, we record the chain of size-change graphs (corresponding to the chain of calls) which lead to the cycle. Such a chain of size-change graphs constitute a \emph{multipath}. To identify whether infinite descent is present in a multipath, it is collapsed to a single size-change graph through a simple composition procedure\,\citep{LeeJones01SizeChange}. If some size-change arc represents a \emph{descending} relationship in such a collapsed size-change graph, then the recursive call leads to termination. If all recursive calls in a program leads to termination, then the entire program is \emph{size-change terminating}.

The size-change termination checker does not cover coinductive definitions. Due to lack of time, the productivity algorithm presented in Section~\ref{sec:productivity} has not been implemented in \textit{findus}.

\subsection{Reflections on Implementation}
From exploring the implementation of codata and copatterns we have found that implementing them is similar to implementing recursive types. In contrast to inductive types, coinductive types do not have to be rolled or unrolled as they are defined by observations, not constructors. We roll to construct inductive types and unroll to inspect how an inductive type was constructed. Neither of these notions make sense in terms of coinductive types. We define coinductive types by giving the observations that can be made on them, and we inspect them by actually observing them. This makes type checking fairly straightforward. We have also discussed evaluation, which is fairly simple with copatterns due to unfolding. With this in mind, the challenge of implementing copatterns does not lie in the implementation itself, but rather in (1) implementing a productivity checker as discussed in Section~\ref{sec:productivity}, and (2) how they can be incorporated into Idris, which will be discussed in Section~\ref{sec:copatterns_in_idris}.
%!TEX root = ../main.tex
\section{Checking the Productivity of Corecursive Functions}
\label{sec:productivity}

Before delving into a description of our proposal for a productivity checking algorithm for definitions with copatterns, we will discuss the properties of other approaches, and why they may or may not be desirable.

\subsection{Lazy Evaluation: The Haskell Approach}
It may be compelling to think that we can simply model coinductive data in the same way as it is done in Haskell, by evaluating all expressions lazily. Then any productivity algorithm would work for Haskell as well, making the result more generally applicable. The lazy evaluation approach, however, is insufficient in several aspects.

First and foremost, if we insist on making no distinction between inductive and coinductive data in our type system (Haskell makes no distinction), subject reduction is lost in a dependently typed system\,\citep{Abel13Copatterns}. The problem lies in the fact that coinductive data is modeled as the \emph{construction} of infinite trees, making dependent pattern matching possible on codata. Consider the type \texttt{U} in Figure~\ref{fig:subject_reduction_problem}, with inhabitant \texttt{u}. Even though it should not hold, the equality \texttt{u = C u} in the type of \texttt{eqU} holds when we allow dependent pattern matching on \texttt{x} in \texttt{eq}, since it allows the type system to reduce \texttt{x} to \texttt{C y}. In this case the type of \texttt{refl} in \texttt{eq} becomes \texttt{C y = force (C y)}, which should hold. However, if we replace the right-hand side of \texttt{eqU} with \texttt{refl}, a type error occurs since \texttt{u} is not equal to \texttt{C u}. Consequently, the type of \texttt{eq} changes when we do dependent pattern matching on its input, \texttt{x}, which means that subject reduction is lost. This problem is explained in greater detail by Abel et al.\,\citep{Abel13Copatterns}, and is also discussed in a correspondence initiated by Danielsson on the Agda mailing list\,\citep{OuryCounterexample}. Naturally, we want to preserve subject reduction in the Idris type system. When we are able to distinguish between inductive and coinductive data, one solution to this problem is to not allow dependent pattern matching on coinductive data. This approach, which is also implemented in Idris, is discussed in Section~\ref{sec:copatterns_in_idris}.

\begin{figure}
\begin{alltt}
data U : Type where         -- No distinction between data and codata
  C : U -> U

fix : (a -> a) -> a
fix f = f (fix f)

u : U
u = fix u

force : U -> U
force x = case x of
            C y => C y

eq : (x : U) -> x = force x
eq x = case x of
         C y => refl

eqU : u = C u
eqU = eq u
\end{alltt}
\caption{Oury's counterexample\,\citep{OuryCounterexample} in a dependently typed language with Haskell-like syntax. Here, \texttt{=} denotes propositional equality, with the sole constructor \texttt{refl}. Dependent pattern matching happens with \texttt{case} expressions.}
\label{fig:subject_reduction_problem}
\end{figure}

Another argument against the Haskell approach is that the core type theory underlying Idris is intended to have the Church-Rosser property\,\citep{BradyIdrisImpl13}, meaning that distinct reduction strategies lead to the same normal form. Furthermore, the total part of Idris is intended to be strongly normalizing, such that it enjoys the \emph{strong} Church-Rosser property\,\citep{Turner04totalfunctional}. This means that not only will every reduction strategy leading to a normal form lead to the same normal form, but \emph{any} reduction strategy must lead to a normal form. In Haskell, many definitions will lead to a normal form under lazy evaluation, while eager evaluation would lead to infinite recursion. Therefore, exploiting lazy evaluation in the total part of Idris would mean that it would no longer have the strong Church-Rosser property.

\subsection{Productivity Checking with Sized Types}
As shown by Abel and Pientka\,\citep{Abel13Wellfounded}, verifying the productivity of coinductive definitions with copatterns is indeed possible using sized types. Thus, the main argument against using sized types is one of usability, from the point of view of the Idris user. Thibodeau\,\citep{Thibodeau11} emphasizes that size annotations generally make the code harder to read and seem unnecessary (see Section~\ref{sec:related_work}). In our own experience, size annotations place quite a burden of bookkeeping upon the user, as they have a tendency to become a tool that is used to satisfy the productivity checker, instead of leading to clearer type specifications. Because totality is optional in Idris for definitions that to not appear in types (as opposed to in Agda, for instance), this means that in a practical programming setting, some users would most likely be inclined to switch off productivity checking for coinductive definitions, or refrain from using them entirely.

This argument is only valid as long as we have no way to fully reconstruct all size annotations. Once we do (if ever), the burden of sized types can be placed entirely upon the compiler. At such point, totality checking with sized types would become a practical approach.

\subsection{Guarded Corecursion}
\label{sec:guarded_cored}
In its original form presented by Coquand\,\citep{Coquand94} and Gim\'{e}nez\,\citep{Gimenez95}, the guardedness condition is generally too restrictive, as discussed in Section~\ref{sec:related_work}. The work done on the ESFP system by Telford and Turner\,\citep{Telford97ensuringstreams,Telford98ensuringthe} makes the guardedness criterion more generally applicable, even though some problems still remain, such as handling indirect application of corecursive functions. An example of this problem is the \texttt{g} function in Figure~\ref{fig:TelfordTurnerProblems}, the guardedness of which cannot be determined due to the indirect call to another function. In this case, this other function is the identity function, but the point is that the result of \texttt{(fst~funPair)} could have been any function. The function \texttt{f} is not considered guarded within their system because forward references may not be made to the rest of the process, i.e. the head of \texttt{f} cannot refer to the tail of \texttt{f}.

\begin{figure}
\begin{alltt}
-- f = 1 :: 1 :: \ldots                            -- g = 1 :: 1 :: \ldots
f = (head (tail f)) :: (1 :: f)                 g = 1 :: ((fst funPair) g)
                                                where funPair = (id, id)
\end{alltt}
\caption{Two types of corecursive functions not deemed productive by the extended guardedness criterion by Telford and Turner\,\citep[Section~6.3]{Telford98ensuringthe}. Here, \texttt{::} is the cons operator for a hypothetical definition of streams without copatterns, and \texttt{id} is the identity function. The functions \texttt{head} and \texttt{tail} are defined by pattern matching.} 
\label{fig:TelfordTurnerProblems}
\end{figure}

Abel\,\citep{Abel99terminationchecking} argues that termination analysis should abstract from the syntactic structure of the program being analyzed to avoid making the outcome of the analysis susceptible to small changes in program structure. While this is true, non-syntactic checks can require the user to pay more attention to the productivity checker, which might not always be advantageous. 

\paragraph{}
What we ultimately seek is a productivity checking algorithm for definitions with copatterns that covers as many realistic cases as possible, without burdening the user too much. In the following, we will provide a detailed description of our proposal.
%An explanation of our proposal for productivity checking in collaboration with size-change termination.
% Why not guardedness?
% Why not sized types? 
% Why not the haskell way?
\subsection{Proposed Solution}
The basic idea of our proposal is \textit{time measurement}. We wish to always know how many observations we can safely make on a given recursive reference. To do this we use time as an analogy. When defining what happens at a given time we can only refer to what has happened earlier. Each observation is defined by what happens one time unit later. We will use days as an analogy. To know what happens on any given day we can only refer to what has happened earlier than that day. If we translate the days to numbers, where later days are given higher numbers, we can express this idea of only being able to reference back in time with an inequality.

If a given function \texttt{f} ``is defined'' on day $\tau$, then the first observation on \texttt{f} is defined on day $\tau+1$, the second observation on day $\tau+2$ etc. If we wish to define what is defined on any given day $\tau+n$ we can only refer to observations made earlier, on an earlier day $\phi$. We know that $\phi$ is an earlier day if $\tau+n\, \textgreater \,\phi$ holds. Any definition that comply with this is said to be time consistent, as there are no references to something that has not yet happened. If we think of observations as places in time, if we cannot refer to something that has not yet happened, we cannot refer to an observation that does not exist. Therefore, any observation that is time consistent is also safe. As such we can see time consistency as equivalent to productivity.

In Figure~\ref{fig:zeros} we decorate a definition \texttt{zeros} with these time measures. Since \texttt{zeros} is the starting point we give that a base measure $\tau$. Both the \texttt{head} and the \texttt{tail} observations are defined one day later than \texttt{zeros}, they both get time measure $\tau+1$. To determine if \texttt{zeros} is productive we must examine the right hand sides of the observations. The \texttt{head} observation is simple as there are no corecursive calls. Therefore it gets the time measure $stable$ as we assume that \texttt{Z} is always known. We use this measure for any reference that is not corecursive. In terms of an numeric value $stable$ can be thought of as negative infinity. In the \texttt{tail} observation we have a corecursive reference to \texttt{zeros}. This gets the time measure $\tau$ which is the time of \texttt{zeros}. To determine the productivity we must ensure that observations are only defined by what has already happened. Looking at this as an inequality we can see that \texttt{zeros} is productive because in the \texttt{head} case $\tau+1\,\textgreater\,stable$ and in the \texttt{tail} $\tau+1\,\textgreater\,\tau$. Note that all these annotations can be inferred.

\begin{figure}
\begin{tabular}{l c}

\begin{minipage}{3in}
\begin{Verbatim}[commandchars=\\\{\},codes={\catcode`$=3\catcode`_=8}]
zeros : Stream Nat
head zeros = Z
tail zeros = zeros
\end{Verbatim}
\end{minipage} &
\begin{minipage}{3in}
\begin{Verbatim}[commandchars=\\\{\},codes={\catcode`$=3\catcode`_=8}]
zeros$_{\tau}$ : Stream Nat
head$_{\tau+1}$ zeros$_{\tau}$ = Z$_{stable}$
tail$_{\tau+1}$ zeros$_{\tau}$ = zeros$_{\tau}$
\end{Verbatim}
\end{minipage}

\end{tabular}
\caption{An infinite stream of zeros. To the left is the implementation, and to the right is the implementation decorated with time measures.}
\label{fig:zeros}
\end{figure}

In Figure~\ref{fig:zerosprime} we see a diverging implementation of \texttt{zeros} called \texttt{zeros'}. To establish that this is not productive we decorate the program in the same fashion as above, but this time the recursive reference in the \texttt{tail} case has time measure $\tau+1$ rather than $\tau$ because we make an observation on the right-hand side of \texttt{zeros}. For this to be productive, $\tau+1\,\textgreater\,\tau+1$ must hold which it clearly does not. Therefore we can say that \texttt{zeros'} is not productive.

\begin{figure}
\begin{tabular}{l c}

\begin{minipage}{3in}
\begin{Verbatim}[commandchars=\\\{\},codes={\catcode`$=3\catcode`_=8}]
zeros' : Stream Nat
head zeros' = Z
tail zeros' = tail zeros'
\end{Verbatim}
\end{minipage} &
\begin{minipage}{3in}
\begin{Verbatim}[commandchars=\\\{\},codes={\catcode`$=3\catcode`_=8}]
zeros'$_{\tau}$ : Stream Nat
head$_{\tau+1}$ zeros'$_{\tau}$ = Z$_{stable}$
tail$_{\tau+1}$ zeros'$_{\tau}$ = tail$_{\tau+1}$ zeros'$_{\tau}$
\end{Verbatim}
\end{minipage}

\end{tabular}
\caption{A non-productive implementation of \texttt{zeros}.}
\label{fig:zerosprime}
\end{figure}

\subsubsection{Coinductive Definitions with Coinductive Parameters}

Sometimes we want to write definitions that depend on other functions with coinductive parameters. When decorating \texttt{nats} in Figure~\ref{fig:nats_productivity} we can not give the right hand side of the \texttt{tail} observation an accurate time measure as we don't know what \texttt{map} does to \texttt{nats}. We must first analyse \texttt{map} to see the effect it has on its parameters.

\begin{figure}
\begin{Verbatim}[commandchars=\\\{\},codes={\catcode`$=3\catcode`_=8}]
nats$_{\tau}$ : Stream Nat
head$_{\tau+1}$ nats$_{\tau}$ = Z$_{stable}$
tail$_{\tau+1}$ nats$_{\tau}$ = (map S nats)$_{?}$
\end{Verbatim}
\caption{A \texttt{Stream} of all the natural numbers, annotated with time measures.}
\label{fig:nats_productivity}
\end{figure}

When analysing \texttt{map} we introduce a new time measure for each coinductive input the function takes. This measure is not used for checking the productivity \texttt{map} but is used for callers of \texttt{map} to see how many observations \texttt{map} performs on a given input. This means that in addition to checking the productivity of \texttt{map} we also create a \textit{specification} for \texttt{map} that other functions can then use, to determine whether the call to \texttt{map} would be productive or not. A specification can be saved in an internal hidden type for later use by the productivity checker.

\begin{figure}
\begin{Verbatim}[commandchars=\\\{\},codes={\catcode`$=3\catcode`_=8}]
map$_{\phi, \upsilon}$ : (a -> b) -> Stream a -> Stream b
head$_{\phi+1}$ (map f s)$_{\phi}$ = f (head$_{\upsilon+1}$ s$_{\upsilon}$)$_{stable}$
tail$_{\phi+1}$ (map f s)$_{\phi}$ = map f (tail$_{\upsilon+1}$ s$_{\upsilon}$)$_{\phi}$
\end{Verbatim}
\caption{The map function for Streams.}
\label{fig:map}
\end{figure}

Making the specification is fairly simple. Since we are interested in knowing how many new observations \texttt{map} possibly makes on an input we are looking for the worst case for every input. In the case of \texttt{map} the worst case for time measure $\upsilon$ is $\upsilon+1$, which tells us that \texttt{map} performs at most one observation on the input. The specification, however, is not done. While it is true that \texttt{map} does make one observation on its input, it does not do so until \texttt{map} itself is observed. This means that any reference to \texttt{map} would need one more observation on its left-hand side to invoke the observations \texttt{map} makes. So even though \texttt{map} adds one observation, it does so one day later. This delay of the observation counteracts the observation map makes. We model this counteraction by subtracting one from the time measure given to $\upsilon$ in the specification.

This gives us $\upsilon+1-1$ or just $\upsilon$, which means that the specification for \texttt{map} does not perform any additional observations on its input. This means we can now complete the annotation of \texttt{nats} from Figure~\ref{fig:nats_productivity}. Since the specification of \texttt{map} tells us that any input remains unchanged, we know that the time measure of the call to \texttt{map} with \texttt{nats} as input in Figure~\ref{fig:natsComplete} is $\tau$. We can therefore conclude that \texttt{nats} is productive, as $\tau+1\,\textgreater\,\tau$.

\begin{figure}
\begin{Verbatim}[commandchars=\\\{\},codes={\catcode`$=3\catcode`_=8}]
nats$_{\tau}$ : Stream Nat
head$_{\tau+1}$ nats$_{\tau}$ = Z$_{stable}$
tail$_{\tau+1}$ nats$_{\tau}$ = (map S nats$_{\tau}$)$_{\tau}$
\end{Verbatim}
\caption{A \texttt{Stream} of all the natural numbers with full time annotations.}
\label{fig:natsComplete}
\end{figure}

\subsubsection{Mutual Recursion}
For checking productivity of mutually recursive coinductive definitions, we assume that we know which definitions refer to each other. Similar to the previous example we then make a specification for each function that can then be used to check the productivity of the callers. Instead of this specification saying what happens to the input, it says what happens to a specific call. Looking at two mutually recursive definitions \texttt{f} and \texttt{g} in Figure~\ref{fig:mutRec1}. For \texttt{f} we define specification $\tau_{g}$ saying what happens to the time measure $\tau$ in \texttt{g} and vice versa we define $\upsilon_{f}$ for \texttt{g}. For the \texttt{tail} case in \texttt{f} to be productive we need that $\tau+1\,\textgreater\,\tau_{g}$ and for \texttt{g} we need $\upsilon+1\,\textgreater\,\upsilon_{f}+1$ to hold.

\begin{figure}
\begin{Verbatim}[commandchars=\\\{\},codes={\catcode`$=3\catcode`_=8}]
f$_{\tau,\upsilon_{f}}$ : Stream Nat
head$_{\tau+1}$ f$_{\tau}$ = Z$_{stable}$
tail$_{\tau+1}$ f$_{\tau}$ = g$_{\tau_{g}}$

g$_{\upsilon,\tau_{g}}$  : Stream Nat
head$_{\upsilon+1}$ g$_{\upsilon}$ = Z$_{stable}$
tail$_{\upsilon+1}$ g$_{\upsilon}$ = tail$_{\upsilon_{f}+1}$ f$_{\upsilon_{f}}$
\end{Verbatim}
\caption{Two mutually corecursive definitions.}
\label{fig:mutRec1}
\end{figure}

Creating these specifications again is simple, and inferrable. To find $\upsilon_{f}$ we look at the references to \texttt{g} in \texttt{f}. We see that there is one observation on the left hand side and none on the right hand side. For the same reason as with \texttt{map} we subtract one for the left hand observations and add one for the right hand side observations. This means that $\upsilon_{f} = \upsilon - 1$. Using the same approach for finding $\tau_{g}$ we get that $\tau_{g} = \tau$. We can now substitute these values into the inequalities from earlier and get that $\tau+1\,\textgreater\,\tau$ and $\upsilon+1\,\textgreater\,\upsilon - 1$. Therefore \texttt{f} and \texttt{g} are productive.

\subsubsection{Known Limitations}
We have discovered a few limitations to this approach. In this section we will present them and discuss possible solutions.

Our approach does not solve the problems discussed in Section~\ref{sec:guarded_cored} directly. The first is if an observation refers to something that is trivially productive, but under more observations. Consider the example in Figure~\ref{fig:forwardRef}. This definition is productive, but will be discarded by our approach as it does not hold that $\tau+1\,\textgreater\,\tau+2$. It is, however, easily solvable due to our use of copatterns. Since \texttt{head tail h} is directly defined we can say that \texttt{head h} is productive if \texttt{head tail h} is productive.

\begin{figure}
\begin{Verbatim}[commandchars=\\\{\},codes={\catcode`$=3\catcode`_=8}]
h$_{\tau}$ : Stream Nat
head$_{\tau+1}$ h$_{\tau}$         = head$_{\tau+2}$ tail$_{\tau+1}$ h$_{\tau}$
head$_{\tau+2}$ tail$_{\tau+1}$ h$_{\tau}$ = Z$_{stable}$
tail$_{\tau+2}$ tail$_{\tau+1}$ h$_{\tau}$ = tail$_{\tau+1}$ h$_{\tau}$
\end{Verbatim}
\caption{An example of a productive definition our productivity checker would discard.}
\label{fig:forwardRef}
\end{figure}

Another limitation is functions returning arbitrary functions. In Figure~\ref{fig:funList} we cannot give an accurate time measure to \texttt{(funPair) i} because we do not know anything about what \texttt{fst funPair} does to \texttt{i}. Attempting to solve this could be done by estimating a worst case scenario for \texttt{funPair}. This, however, involves a lot of unfolding risking making the productivity checker very slow, while still not providing a very accurate analysis. We have not been able to find a desirable solution to this problem.

\begin{figure}
\begin{Verbatim}[commandchars=\\\{\},codes={\catcode`$=3\catcode`_=8}]
i$_{\tau}$ : Stream Nat
head$_{\tau+1}$ (i)$_{\tau}$ = Z$_{\tau-1}$
tail$_{\tau+1}$ (i)$_{\tau}$ = (fst funPair) i$_{?}$

funPair : (Stream a, Stream a)
funPair = (id, id)
\end{Verbatim}
\caption{An example of a productive definition our productivity checker would discard.}
\label{fig:funList}
\end{figure}

\subsubsection{Possible Algorithm}
We are convinced that time consistency is determinable by a machine. All analysis is syntactical which means that the user of the language does not have to annotate anything. To support this claim we have made a natural language description of a possible algorithm. 

We consider a specification a function taking a time measure and producing a new time measure reflecting the changes noted by the specification. Also, note that destructors also can be considered coinductive functions with a specification saying that it increases the time by one. For ease of reading we have defined the following function which will be refered to several times as \textit{calculate}.

\paragraph{
Given a coinductive definition with time measure $\phi$, if the definition is applied to any coinductive functions, apply the specification of these functions to $\phi$ and return the result. If it is not applied to any coinductive functions, just return $\phi$.}
\paragraph{f: Coinductive Definition}
\begin{enumerate}
\item Create a time measure $\tau$ for the definition.
\item For each observation in the definition do:
\begin{enumerate}
\item Count the number of left-hand destructors and add this number to $\tau$. This is the time of this observation. 
\item Find all corecursive reference on the right-hand side. Calculate the time measure of all these references using \textit{calculate}.
\item Find all non-corecursive reference on the right-hand side. Assign them the time measure $stable$.
\item The time of the observation (from (a)) must be greater than the greatest time measure on the right-hand side.
\end{enumerate}
\item If 2.(d) holds for all observations the definition is productive.
\end{enumerate}

\paragraph{g: Coinductive Functions}
\begin{enumerate}
\item Check the productivity of function itself using $f$, ignoring the coinductive parameters.
\item For each input parameter create a time measure.
\item For each observation in the definition do:
\begin{enumerate}
\item Count the number of left-hand destructors. This is the time offset.
\item For each reference to the coinductive input do:
\begin{enumerate}
\item Calculate the time measure of the reference using \textit{calculate}.
\item Subtract the offset from these the result.
\end{enumerate}
\end{enumerate}
\item For each input find the greatest time measure in all observation.
\item These greatest measures is the specification.
\end{enumerate}
\paragraph{h: Mutually Corecursive Definition}
\begin{enumerate}
\item For each mutually corecursive definition do:
\begin{enumerate}
\item Check the productivity of the function itself using $f$, ignoring the mutually corecursive references
\item Create a time measure for every definition with which it is mutually corecursive.
\item For each observation with a reference to a mutual definition do:
\begin{enumerate}
\item Count the number of left-hand destructors. This is the time offset.
\item Calculate the time measure of the references using $calculate$.
\item Subtract the offset from these references.
\end{enumerate}
\item For each definition this definitiion with which it is mutually corecursive, find the greatest time measure. These are the time measures of other definitions in this definition.
\item For each observation with a reference to a mutual definition do:
\begin{enumerate}
\item Count the number of left-hand side destructors. This is the time of this observation.
\item Look up the time measure for this definition in that definition.
\item Calculate the time measure of this reference using $calculate$ with the time measure from above as input.
\item The time of the observation must be greater than the greatest time measure given on the right-hand side.
\end{enumerate}
\item If the 1.e.iv. holds for all observation this definition is productive.
\end{enumerate}
\end{enumerate}

%!TEX root = ../main.tex
\section{Copatterns in Idris}
\label{sec:copatterns_in_idris}
In this section we will describe the internal changes needed, from a high-level perspective, for adding coinductive definitions by obserations and copatterns to Idris.

\subsection{Representation}
To add definitions with copatterns to Idris, we will need to add two high-level constructs to the language: (1) definitions of codata types in terms of observations, and (2) a way of defining corecursive functions using copatterns. We envision that the resulting Idris constructs will make definitions such as those represented in Figure \ref{fig:envisioned_copattern_syntax} possible.

\begin{figure}
\begin{alltt}
codata Stream : Type -> Type where             repeat : a -> Stream a
  head : Stream a -> a                         head (repeat a) = a
  tail : Stream a -> Stream a                  tail (repeat a) = repeat a
\end{alltt}
\caption{Examples of the envisioned Idris syntax. On the left, a codata defintion for streams with two observations, on the right a corecursive function using this stream definition.}
\label{fig:envisioned_copattern_syntax}
\end{figure}

Several changes will have to be made to the Idris compiler in order to achieve this. To identify where these changes must be made, let us first examine its high-level phases, shown in Figure \ref{fig:Idris_compiler_phases}.

\begin{figure}
\begin{center}
$\mbox{\Idris{}}$
$\;$
$\xrightarrow{\mathrm{ (desugaring) }}$
$\;$
$\mbox{\IdrisM{}}$
$\;$
$\xrightarrow{\mathrm{ (elaboration) }}$
$\;$
$\mbox{\TT{}}$
$\;$
$\xrightarrow{\mathrm{ (compilation) }}$
$\;$
$\mbox{Executable}$
\end{center}
\caption{The phases of the Idris compiler. Borrowed from Brady\,\citep{BradyIdrisImpl13}.}
\label{fig:Idris_compiler_phases}
\end{figure}

\begin{itemize}
\item \Idris{} is the high-level language with which the user interacts.
\item \IdrisM{} % see section 4.5
is a desugared subset of Idris. It is ``desugared'' in the sense that it does not have \texttt{do}-notation or infix operators, and because all implicit arguments from Idris are explicitly bound.
\item \TT, The core Type Theory (hence \TT), is a dependently typed lambda calculus extended with pattern matching and inductive families (corresponding to \texttt{data} declarations in Idris). The philosophy behind having a core type theory is that if everything is compiled to a small core, then we only have to place our trust in the correctness of the core, instead of the entire system. Therefore, we do not wish to change anything in \TT{} unless absolutely necessary.
\end{itemize}

\subsubsection{High-level Changes: \Idris{} and \IdrisM{}}
In order to transform the source text of our program with copatterns into (the abstract syntax of) Idris, we have to change the parser accordingly. Specifically, we will need it to accept definitions such as those shown Figure \ref{fig:envisioned_copattern_syntax}. To achieve this, the abstract syntax must also be modified to accomodate these changes. In the desugaring phase, we will need to make sure that implicit parameters in any \texttt{codata} definition are explicitly bound, and that the right-hand sides of any corecursive function with copatterns are desugared in the same way as the rest of the program.

\subsubsection{Changes to the Elaborator}
The elaborator translates high-level Idris programs into \TT{} by using these high-level programs to direct tactic proofs that correspond to equivalent \TT{} programs. We have yet to determine exactly how \texttt{codata} and corecursive functions with copatterns should be elaborated. If coinductive types can be represented directly in \TT, then we will have to extend the elaborator to handle programming with observations. If not, we will most likely transform our new constructs into existing Idris constructs as described in the next subsection, and use the already existing elaboration rules for these.


\subsubsection{Changes to the Core Type Theory}
We argue that it will not be necessary to make any changes to the core type theory. This argument is twofold. First, we argue that it is possible to reduce \texttt{codata} definitions with observations to \TT, and second, we make the same argument for corecursive functions defined with copatterns.

(1) For the existing definition of codata in Idris (see Section \ref{sec:background}), delay and force semantics\,\citep[Section 3.5]{Abelson96SICP} are already implemented to facilitate lazy evaluation. Any coinductive argument to a \texttt{codata} constructor is delayed at the time of construction, and implicitly forced during pattern matching. These semantics can also be used for defining coinductive data by observations. Let us formalize this notion by considering the \texttt{Inf} data type given in Figure \ref{fig:Inf_datatype}. In essence, this type enables us to create promises that some data will be delivered at some time in the future using the \texttt{Delay} constructor, while the \texttt{Force} constructor allows us to fulfill such a promise. The name reflects that this data type allows us to model infinite data structures.

\begin{figure}
\begin{alltt}
data Inf : Type -> Type where
  Delay : a -> Inf a
  Force : Inf a -> a
\end{alltt}
\caption{A data type modeling the delay and force semantics. This is a simplified version of the actual Idris semantics.}
\label{fig:Inf_datatype}
\end{figure}

Using \texttt{Inf} with the current \texttt{codata} mechanics in Idris, we can arrive at the definition of streams shown in Figure \ref{fig:stream_current_Inf} (the same as was shown in Figure~\ref{fig:stream_current_Inf_and_natsFrom}). In this case, the type of the \texttt{(::)} constructor tells us that any stream consists of some data (the first argument, \texttt{a}) and a promise to deliver the rest of the data later (the second argument, \texttt{Inf (Stream a)}).

\begin{figure}
\begin{alltt}
codata Stream : Type -> Type where
  (::) : a -> Inf (Stream a) -> Stream a
\end{alltt}
\caption{A stream definition as it currently looks in Idris using the \texttt{Inf} data type.}
\label{fig:stream_current_Inf}
\end{figure}

Definitions of codata by observations can be created in a similar way, only that the logic must be flipped. A definition of streams by observations using the \texttt{Inf} data type is presented in Figure \ref{fig:stream_copatterns_Inf}. This definition says that if we observe the \texttt{head} of the stream, we get some data, and if we observe the \texttt{tail} of the stream, we get a promise that the rest of the stream can be computed later.

\begin{figure}
\begin{alltt}
codata Stream : Type -> Type
  head : Stream a -> a
  tail : Stream a -> Inf (Stream a)
\end{alltt}
\caption{A stream definition by observations using the \texttt{Inf} data type.}
\label{fig:stream_copatterns_Inf}
\end{figure}

Examining the types of the constructor in Figure \ref{fig:stream_current_Inf} and the observations in Figure \ref{fig:stream_copatterns_Inf}, we observe that there is a close correspondence: the types of the parameters to \texttt{(::)} exactly match the resulting types of the \texttt{head} and \texttt{tail} observations. We argue that it should be possible to model the definition in Figure \ref{fig:stream_copatterns_Inf} using the definition in Figure \ref{fig:stream_current_Inf} by transforming the two observations to one constructor with two parameters. To simulate the act of observing a stream, we can create two auxiliary functions, which can provide us with the head and tail of the stream through pattern matching. In fact, we argue that for any definition by observations, such a transformation is possible. If some coinductive type \texttt{A} is defined by \emph{n} observations, we can transform it into a definition by constructors by creating one constructor for \texttt{A} with \emph{n} parameters, and creating \emph{n} auxiliary functions to simulate each observation. Since this means that codata definitions by observations can be modeled using only exising Idris constructs, we expect that no changes in the core type theory will be necessary to enable such definitions.

(2) While receiving promises might be fine, a time must come when such promises must be fulfilled if they are to have any value at all. In the current Idris implementation, this is done by implicitly forcing during pattern matching. Consider the (rather contrived) example of the \texttt{headtail} function in Figure \ref{fig:headtail}, which returns the second element of a stream (as presented in Figure~\ref{fig:stream_current_Inf}). Here, we see that if we at an earlier time were promised to get \texttt{xs}, we can use \texttt{Force} to command that now is the time to deliver on that promise. Once the computation is done, we receive the head of \texttt{xs}, and a new promise for the computation of the rest of the stream. Note that \texttt{Force} instigates computation of the promise, and thus cannot simply be omitted.

\begin{figure}
\begin{alltt}
headtail : Stream a -> a
headtail (x :: (Delay xs)) = case Force (Delay xs) of
                               (y :: (Delay ys)) => y
\end{alltt}
\caption{An example showing how pattern matching fulfills promises by forcing a delayed tail to be computed. Here, \texttt{Delay} and \texttt{Force} are explicit.}
\label{fig:headtail}
\end{figure}

Again, a similar method can be used for observations by applying copatterns. Whenever we make an observation which returns a promise of a coinductive type,
we can use the \texttt{Force} constructor to force the next step to be computed. In Figure \ref{fig:headtail_copatterns}, we see that the only way that \texttt{headtail'} will get past the type checker is if we apply \texttt{Force} after making the \texttt{tail} observation. Informally, this says: ``force this stream to unfold once, and provide me with the head of the result''. Since the places where \texttt{Force} is needed can be predicted quite mechanically, it should be possible for the elaborator to insert these automatically.

\begin{figure}
\begin{alltt}
headtail' : Stream a -> a
headtail' s = head (Force (tail s))
\end{alltt}
\caption{An example showing how promises can be fulfilled when using copatterns. The \texttt{Stream} definition used here is the one shown in Figure~\ref{fig:stream_copatterns_Inf}.}
\label{fig:headtail_copatterns}
\end{figure}

We have already argued that observations can be modeled by auxiliary functions and pattern matching. Therefore, there should exist a straightforward transformation from the the \texttt{headtail'} function to the \texttt{headtail} function. Consequently, programming with observations would not require any changes to the core type theory either. As the idea of \texttt{Inf} already exists for the current implementation of codata in Idris, it should be possible to reuse this idea for implementing codata by observations and copatterns.

\subsection{Adding the Productivity Checker}
\label{sec:copattern_in_idris_productivity_checker}

%How can we fit the proposed approach to productivity into Idris.

% Every definition of that needs size-change termination has an equivalent definition without copatterns
% Mutual recursion

In addition to the constructs for copatterns, the productivity checker presented in Section \ref{sec:productivity} must also be incorporated into Idris. A productivity checker based on simple guarded corecursion as described in Section \ref{sec:related_work} already exists for the current codata implementation, but it is quite conservative and not built with copatterns in mind. Therefore, we plan to implemenent the productivity checker for codata definitions by observations from scratch.

\subsubsection{Mutual Recursion}
In order to be able to implement the envisioned algorithm for checking productivity, we have to know which functions that are mutually recursive, as these must be handled specially. Since Idris already requires mutually recursive definitions to be defined within a \texttt{mutual} block, we do not expect this to become a problem.

\subsubsection{Interaction with the Size-Change Termination Checker}
For functions on inductive data, Idris employs a termination checker based on the size-change principle\,\citep{LeeJones01SizeChange}. When functions that mix pattern matching on inductive data and copatterns are encountered, identifying when the productivity checker and the size-change termination checker will need to interact becomes interesting.

In our experience, most corecursive functions which (1) are defined using copatterns, and (2) rely on the size-change principle for totality, has an equivalent (and often more concise) definition without copatterns. Consider the example in Figure \ref{fig:drop_drop'}. The \texttt{drop'} function is not productive according to our productivity checker, since we cannot be certain that we are allowed to make any observations on \texttt{drop' n (tail s)} on a right-hand side. Nevertheless, we know that according to the size-change principle, we cannot subtract from the \texttt{Nat} argument forever, and thus the function must be deemed productive. The definition of \texttt{drop'} is perfectly valid, but if we examine our use of copatterns more closely, they really do not add any value. A better definition of the same function is \texttt{drop}, which does not use copatterns and is obviously size-change terminating; in fact, we do not even have to check \texttt{drop} for productivity. 

\begin{figure}
\begin{alltt}
drop' : Nat -> Stream a -> Stream a
head (drop' Z     s) = head s
head (drop' (S n) s) = head (drop' n (tail s))
tail (drop' Z     s) = tail s
tail (drop' (S n) s) = tail (drop' n (tail s))

drop : Nat -> Stream a -> Stream a
drop Z     s = s
drop (S n) s = drop n (tail s)
\end{alltt}
\caption{Two equivalent (and valid) definitions of the drop function. They both rely on size-change termination for totality, so nothing is gained by using copatterns in the upper example, \texttt{drop'}.}
\label{fig:drop_drop'}
\end{figure}

On the surface, the \texttt{prepend} function in Figure \ref{fig:prepend} shows a similar situation as for the \texttt{drop'} function: \texttt{prepend} is size-change terminating on its recursive call. From the previous argument, it is therefore compelling to think that this function should be defined without copatterns. However, since we need to add observations to the input stream, copatterns \emph{must} be used in this example. Furthermore, even though \texttt{prepend} is size-change terminating, it does not matter, since it is also productive.

\begin{figure}
\begin{alltt}
prepend : List a -> Stream a -> Stream a
head (prepend []      s) = head s
head (prepend (x::_)  s) = x
tail (prepend []      s) = tail s
tail (prepend (_::xs) s) = prepend xs s
\end{alltt}
\caption{A function that prepends a list on a stream.}
\label{fig:prepend}
\end{figure}

The point we want to stress here is one of programming style. In general, functions defined using copatterns should not need to rely on size-change termination for productivity. In such cases, it is most likely possible to rewrite the function without copatterns. 

Suppose we are not aware of this hint, and insist on getting our \texttt{drop'} function from Figure \ref{fig:drop_drop'} past the totality checker. In this case, we have to run the function through the size-change termination checker. There are two ways by which we can enable the termination checker to accept corecursive functions with copatterns. The first is to add definitions with copatterns to the set of Idris constructs the termination checker should analyze. Because we have not scrutinized the current implementation of the termination checker, we are not fully convinced that this will be possible, though.

Another possibility is to transform such functions into definitions without copatterns. As we have argued previously, there exists a transformation from any \texttt{codata} type defined by observations (as in Figure \ref{fig:stream_copatterns_Inf}) to one defined by constructors (as in Figure \ref{fig:stream_current_Inf}). Similarly, all cofunctions defined with copatterns can be defined by pattern matching instead. For instance, \texttt{drop'} can be transformed as shown in Figure \ref{fig:drop'_dropT}. Here, \texttt{dropT} is only defined using data constructors and function calls, and is easily discovered to be size-change terminating on its first argument. 

\begin{figure}
\begin{alltt}
drop' : Nat -> Stream a -> Stream a
head (drop' Z     s) = head s
head (drop' (S n) s) = head (drop' n (tail s))
tail (drop' Z     s) = tail s
tail (drop' (S n) s) = tail (drop' n (tail s))

dropT : Nat -> StreamT a -> StreamT a
dropT Z     s = head s :: tail s
dropT (S n) s = dropT n (tail s)
\end{alltt}
\caption{The \texttt{drop'} function (shown again) along with its transformation, \texttt{dropT}. Here, \texttt{StreamT} is equivalent to the stream definition in Figure \ref{fig:stream_current_Inf}, and the \texttt{head} and \texttt{tail} functions are auxiliary functions defined by pattern matching. The force and delay semantics have been elided for clarity.}
\label{fig:drop'_dropT}
\end{figure}

Although we expect to be able to generalize this transformation to work for any function, we hope that it will be possible to simply modify the existing implementation of the size-change termination checker.


%!TEX root = ../main.tex
\section{Future Work}
\label{sec:future_work}
The above sections give rise to several directions for future work. Specifically, we will have to:
\begin{itemize}
\item Formalize the productivity algorithm presented in Section~\ref{sec:productivity}.
\item Show the correctness of the productivity algorithm.
\item Investigate further the internals of the Idris implementation.
\item Find out how codata definitions by observations and corecursive functions with copatterns can be elaborated to \TT.
\end{itemize}
Although the productivity algorithm was described in natural language in Section~\ref{sec:productivity}, a more operational formalization of the algorithm is needed if we are to implement it. A possible step on the way to implementing the algorithm in Idris would be to extend \textit{findus} with dependent types in order to get a realistic testing environment, and then incorporate the productivity algorithm for copatterns into this system. Then, once the algorithm works in this setting, it should be easier to realize in Idris.

To ensure that our productivity algorithm is usable, we will need to prove its correctness. This could possibly be done by reducing the system of time measurement to a subset of sized types.

Since the constructs described in Section~\ref{sec:copatterns_in_idris} must be added to Idris, we will need to delve deeper into the Idris implementation. In particular, we need to investigate how to elaborate these constructs to the core type theory. Additionally, the productivity algorithm must be fit into the system alongside the size-change termination checker.

%!TEX root = ../main.tex
\section{Conclusion}
\label{sec:conclusion}
Throughout this project, our main objective has been to prepare ourselves for for our coming master's thesis, where we plan to implement coinductive data by observations and corecursive function definitions with copatterns in Idris. We have sought to achieve this by first studying the theory behind coinductive data and copatterns. In this process, we found that coinductive data is dual to inductive data, in the sense that where the latter is defined in terms of constructors, the former is most naturally defined in terms of the observations we can make on it. Copatterns provide an elegant way of defining corecursive functions by observations, and thus seem like a valuable addition to Idris. 

To understand the implementation-specific details behind copatterns, we subsequently went ahead and implemented them in a simple functional programming language. From this, we found that neither coinductive types nor copatterns proved to be complex to implement. Coinductive types can be modeled as recursive types (as $\nu$-types, specifically), while copatterns be added as a language construct similar to pattern matching. Therefore, we expect that the greatest challenge ahead will not be the implementation of copatterns in itself, but rather how they can be added to the existing Idris system.

Following an investigation into the existing literature on the size-change principle, we concluded that no implementation of the principle exists which would be suitable as a productivity checking algorithm for definitions with copatterns. When corecursive functions are defined in terms of observations, a different approach than a simple guardedness condition must be used, since the presence of one or more observations on all corecursive occurrences does not imply that a definition is productive. Of the other approaches considered, sized types seem most promising. We decided against a recommendation of sized types for ensuring productivity, however, since it would require the user to provide too many annotations. The solution proposed in Section~\ref{sec:productivity} is thus entirely a syntactic check. The basic idea of the algorithm is that the result of an observation at any time can only be defined in terms of observations that are already known at that time. Non-productive definitions are therefore exactly those definitions which require knowledge about ``the future''.

Concerning Idris, we expect that it will not be necessary to make any changes to the core type theory, since coinductive data defined by observations and function definitions with copatterns can be reduced to already existing constructs. The interplay between the size-change termination checker and our proposed productivity checking algorithm will be interesting, since it is possible for definitions with copatterns to be size-change terminating without being productive according to our algorithm.

We are very optimistic about our future work on implementing these changes in Idris. The solution we propose for productivity checking seems quite robust, although it has a few shortcomings. The fact that it seems unlikely that we will have to change the core type theory in Idris will make the implementation easier, even though other unforeseen complications will possibly occur. In conclusion, we are looking forward to being able to use copatterns in Idris.



\bibliography{bibliography}
\end{document}